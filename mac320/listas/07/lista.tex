\documentclass[12pt]{article}
\usepackage[utf8]{inputenc}
\usepackage[margin=1in]{geometry} 
\usepackage[brazil]{babel}
\usepackage{amsmath,amsthm,amssymb,amsfonts}
 
\newcommand{\N}{\mathbb{N}}
\newcommand{\Z}{\mathbb{Z}}
 
\newcounter{exCounter}
\setcounter{exCounter}{22}
\newtheorem{lema}{Lema}
\newtheorem{ex}[exCounter]{Ex}
\newenvironment{problem}[2][Ex]{\begin{trivlist}
\item[\hskip \labelsep {\bfseries #1}\hskip \labelsep {\bfseries #2.}]}{\end{trivlist}}
\linespread{2}
%If you want to title your bold things something different just make another thing exactly like this but replace "problem" with the name of the thing you want, like theorem or lemma or whatever
 
\begin{document}
 
%\renewcommand{\qedsymbol}{\filledbox}
%Good resources for looking up how to do stuff:
%Binary operators: http://www.access2science.com/latex/Binary.html
%General help: http://en.wikibooks.org/wiki/LaTeX/Mathematics
%Or just google stuff
 
\title{Lista 7}
\author{Victor Sena Molero - 8941317}
\maketitle

\section{Exercícios}
\begin{ex}
Seja $G$ um grafo simples de ordem $n \geq 2k$ e tal que $g(v) \geq k \geq 1$ para todo $v$ em $G$. Mostre que $G$ tem um emparelhamento com pelo menos $k$ arestas.
\end{ex}

\begin{proof}[Prova]
Seja $k$ um inteiro maior que 1 e $G$ um grafo simples de ordem $n \geq 2k$ e tal que $\forall v \in V(G), g(v) \geq k$. Queremos demonstrar a tese de que $G$ tem um emparelhamento $E$ tal que $|E| \geq k$. \\
Primeiro, vamos provar que com $k = 1$, vale a tese. O grafo tem pelo menos uma aresta, pois todo vértice tem grau pelo menos 1, logo, basta escolher uma aresta qualquer e esta será o emparelhamento desejado. \\
Agora, suponha que a tese seja falsa com $k > 1$, então existe um contra exemplo para tal tese. Sabemos que se $G$ for um grafo completo, ele tem um emparelhamento de tamanho $k$, logo, existe um contra-exemplo maximal, ou seja, um grafo tal que não vale a propriedade e, se adicionarmos uma aresta qualquer, a propriedade passa a valer. \\
Seja $G$ tal contra-exemplo maximal. Escolhemos dois vértices não-adjacentes $u$ e $v$ quaisquer de $G$, se adicionarmos a aresta $uv$ gerando o grafo $G'$, criaremos um emparelhamento $E'$ de tamanho $k$. Temos duas opções: \\
Se $uv \notin E'$, então $E'$ é um emparelhamento em $G$, pois só contém arestas que pertencem a $G$. \\
Se $uv \in E'$, então $E' - uv$ é um emparelhamento de tamanho $k-1$ em $G$ onde nem $u$ nem $v$ estão cobertos. Sabemos que $g(u) \geq k$ e $g(v) \geq k$, além disso, $u$ e $v$ não são adjacentes. Mais uma vez, três casos são possíveis: \\
$u$ tem um vizinho $x$ livre, desta forma, basta adicionar a aresta $ux$ ao emparelhamento e gerar um emparelhamento em $G$ de ordem $k$. \\
$v$ tem um vizinho $y$ livre, analogamente, obtemos um emparelhamento de ordem $k$. \\
Caso contrário, todos os vizinhos de $u$ e $v$ estão emparelhados com algum vértice. Suponha, por absurdo, que não existe nenhum par de vértices $x$, $y$ emparelhado em $E'$ onde $u$ é vizinho de $x$ e $v$ é vizinho de $y$. Já que o emparelhamento tem tamanho $k-1$, temos exatamente $2(k-1)$ vértices emparelhados, porém, cada vizinho de $u$ impossibilita um vizinho de $v$ (e vice-versa), logo, deveríamos ter, no mínimo $g(u) + g(v)$ vértices emparelhados, porém $g(u) + g(v) \geq 2k > 2(k-1)$, um absurdo. Logo, existe um par $x$, $y$ onde a aresta $xy$ esta em $E'$ e as arestas $ux$ e $vy$ estão em $G$, ou seja, existe um caminho alterante em $G$ em relação ao emparelhamento $E'$, portanto, podemos formar um novo emparelhamento $E$ em $G$ onde $|E| = |E'| + 1 = k$. \\
Ou seja, em todos os casos possíveis obtivemos um emparelhamento de ordem $k$ no grafo $G$.
\end{proof}

\begin{ex}
Seja $G$ um grafo bipartido com pelo menos uma aresta. Mostre que existe um emparelhamento que cobre todos os vértices de grau $\Delta(G)$.
\end{ex}

\begin{proof}[Prova]
Seja $G$ um grafo $(X,Y)$-bipartido com pelo menos uma aresta. Vamos provar que existe um emparelhamento que cobre todos os vértices de grau $\Delta(G)$. Seja $S$ o conjunto de vértices de grau máximo em $X$ e $T$ o conjunto de vértices de grau máximo em $Y$.
Eu estava tentando achar um caminho alternante entre alguém de grau máximo e alguém que não tivesse grau máximo, mas acho que isso não ajuda :(
\end{proof}

\begin{ex}
Prove que uma árvore tem no máximo um emparelhamento perfeito.
\end{ex}

\begin{proof}[Prova]
Seja $G$ uma árvore qualquer, vamos provar, por indução em $|V(G)|$ que $G$ tem, no máximo, um emparelhamento perfeito. \\
Se $|V(G)| = 0$, $E = \emptyset$ é um emparelhamento perfeito. \\
Com $k > 0$, assumindo que vale a tese com $|V(G)| < k$. \\
Assumindo que $|V(G)| = k$. Se $G$ não contém nenhuma olha, então $k = 1$ e não existe um emparelhamento perfeito. Se $G$ tem uma folha, seja $u$ uma folha qualquer de $G$. \\
O vértice $u$ é adjacente a apenas um outro vértice $v$, portanto, $uv$ pertence a qualquer emparelhamento perfeito de $G$. Seja $H = G - u - v$, $H$ é uma floresta. Nenhuma das arestas diferentes de $uv$ podem pertencer a um emparelhamento perfeito, pois são adjacentes a $uv$, que pertence a todos. \\
A quantidade de emparelhamentos perfeitos em $G$ é igual à quantidade de emparelhamentos perfeitos em $H$, pois um emparelhamento perfeito em $H$ pode ser unido a $uv$ gerando um emparelhamento perfeito em $G$ e se houver um emparelhamento perfeito em $G$ pode-se remover $uv$ dele (que obrigatóriamente pertence a ele) e gerar um emparelhamento perfeito em $H$ (já que nenhuma aresta adjacente a $v$ diferente de $uv$ pode pertencer ao emparelhamento perfeito em $G$). \\
Um emparelhamento perfeito $E$ qualquer de $G$ gera um emparelhamento perfeito em cada uma das árvores $T$ de $H$, pois toda aresta de $E$ diferente de $uv$ pertence a uma árvore de $T$. Logo, pode-se escolher todas as arestas que pertencem à $T$ em $E$, elas obrigatóriamente cobrem todas as arestas de $T$, pois, se não, $G$ não seria um emparelhamento perfeito, logo, elas formam um emparelhamento perfeito em $T$. \\
Por hipótese de indução, toda árvore de $H$ tem no máximo um emparelhamento perfeito. Podemos separar em dois casos: \\
Em um caso, existe pelo menos uma árvore $T$ de $H$ que não tem nenhum emparelhamento perfeito. Suponha, por absurdo, que $G$ tem um emparelhamento perfeito, como provado, ele gera um emparelhamento perfeito em toda árvore de $H$, logo, há um emparelhamento perfeito em qualquer árvore de $H$, um absurdo. Portanto, se existe uma árvore de $H$ sem emparelhamento, não existe emparelhamento perfeito em $G$. \\
No outro caso, toda árvore de $H$ tem exatamente um emparelhamento perfeito. Então, $G$ tem pelo menos um emparelhamento perfeito, basta unir os os emparelhamentos de todas as árvores de $H$ à aresta $uv$, isso cobre todos os vértices de $G$ e gera um emparelhamento perfeito. Por outro lado, $G$ não tem dois emparelhamentos distintos. Seja $E$ o emparelhamento descrito pela união dos emparelhamentos das árvores de $H$ à aresta $uv$. Suponha, por absurdo, que $G$ contém outro emparelhamento perfeito $E'$ tal que $E \neq E'$. Assim, existe pelo menos uma aresta $\alpha$ em $E$ que não pertence a $E'$, seja $T$ a árvore que contém $\alpha$, sabemos que $E'$ gera um emparelhamento perfeito em $T$ que contém $\alpha$, um absurdo pois $T$ tem exatamente um emparelhamento perfeito e ele só contém arestas de $E$, ou seja, não contém $\alpha$.
Ou seja, nos dois casos, ou $G$ não tem emparelhamentos perfeitos, ou $G$ tem exatamente um emparelhamento perfeito. Logo, está provada a tese.
\end{proof}
=======
Primeiro, vamos provar que se $G$ é um grafo simples de ordem $n \geq 2$ e tal que $g(v) \geq 1$ para todo $v$ em $G$, $G$ tem um emparelhamento com pelo menos 1 aresta. Trivialmente, $G$ tem pelo menos uma aresta pois tem pelo menos um vértice e todo vértice tem grau pelo menos 1, logo, podemos escolher esta aresta qualquer e formar um emparelhamento em $G$ de tamanho 1. Agora, só precisamos resolver o exercício no caso $k > 1$.

Seja $k$ um inteiro positivo maior que  1 e $G$ um grafo simples de ordem $n \geq 2k$ tal que $g(v) \geq k \forall v \in V(G)$. Vamos mostrar que $G$ tem um emparelhamento com pelo menos $k$ arestas.

Se a tese for falsa, teremos um contra-exemplo. Além disso, sabemos que um grafo completo de ordem $n \geq 2k$ tem um emparelhamento de tamanho $k$, logo, temos pelo menos um contra-exemplo diferente do grafo completo. Escolhemos um grafo $G$ não-completo onde valem as hipóteses, não vale a tese e tal que, se adicionarmos uma aresta qualquer ao grafo, vale a tese, ou seja, um contra-exemplo maximal.

Escolhemos então um par qualquer de vértices não adjacentes $u$ e $v$ tal que, se adicionarmos a aresta $uv$ ao grafo $G$ formaremos o grafo $G'$ com um emparelhamento $E'$ de tamanho $k$. Se $uv \notin E'$, então $E'$ é também um emparelhamento em $G$, portanto, $G$ tem um emparelhamento de tamanho $k$.

Caso contrário, escolhemos o emparelhamento $E = E' - uv$, sabemos que $|E| = k-1$ e $u$ e $v$ são não-adjacentes. Seja $X$ o conjunto de vértices adjacentes a $u$ e $Y$ o conjunto de vértices adjacentes a $v$, sabemos que $|X| \geq k$ e $|Y| \geq k$. Suponha, por absurdo que não existe nenhum vértice livre em $X \cup Y$ e não existe alguém em $X$ emparelhado com alguém de $Y$, assim, podemos definir o conjunto $Z$ de vértices emparelhados com vértices de $X$, sabemos que $|X| = |Z|$ e que $Z \cap Y = \emptyset$, então temos que $|Z \cup Y| = |Z| + |Y| = |X| + |Y| \geq 2*k$, porém, só existem $2(k-1)$ vértices emparelhados no grafo, então existe alguém em $Z \cup Y$ não emparelhado, ou seja, livre, um absurdo. Assim, sabemos que existem 3 possibilidades (não disjuntas):

\begin{itemize}
    \item Existe alguém livre em $X$, assim, adicionamos uma aresta de $u$ para tal vértice livre no emparelhamento $E'$ gerando um emparelhamento $E$ de tamanho $k$.
    \item O caso com alguém, livre em $Y$ é análogo.
    \item Existe uma aresta emparelhada $xy$ tal que $x \in X$ e $y \in Y$, assim, existe um caminho alternante $u, x, y, v$ no grafo, que tem duas pontas livres, assim, pode-se gerar um emparelhamento de tamanho $k$.
\end{itemize}

Em todos os casos, então, obtivemos um emparelhamento de tamanho $k$.
\end{proof}

\begin{ex}
Seja $G$ um grafo bipartido com pelo menos uma aresta. Mostre que existe um emparelhamento que cobre todos os vértices de grau $\Delta(G)$
\end{ex}

\begin{proof}
Seja $G$ um grafo $(X,Y)$-bipartido com pelo menos uma aresta. Seja $X^*$ o conjunto de vértices de grau $\Delta(G)$ em $X$. Vamos provar que sempre existe um emparelhamento em $G$ que cobre $X^*$.

Suponha, por absurdo, que existe um subconjunto $S$ de $X^*$ tal que $|Adj(S)| < |S|$. Consideremos o grafo $H$ induzido em $G$ por $S \cup Adj(S)$. Já que $H$ é $(S,Adj(S))$-bipartido, temos que
$$ \sum_{u \in S} g_H(u) = \sum_{v \in Adj(S)} g_H(v) $$, mas
$$ \sum_{u \in S} g_H(u) = |S| \Delta(G) $$ e
$$ \sum_{v \in Adj(S)} \leq |Adj(S)| \Delta(G) < |S| \Delta(G) $$, ou seja
$$ |S| \Delta(G) < |S| \Delta(G) $$, um absurdo.

Temos, então, que para todo subconjunto $S$ de $X^*$, $|Adj(S)| \geq |S|$, portanto, vale o teorema de Hall e existe um emparelhamento em $G$ que cobre $X^*$. 

Analogamente, existe um emparelhamento em $G$ que cobre o conjunto $Y^*$ de vértices de grau $\Delta(G)$ em $Y$.

Escolhemos então um emparelhamento $E_X$ que cobre $X^*$ e um $E_Y$ que cobre $Y^*$. Escolha o conjunto de arestas $E = E_X \cup E_Y$, se não houver nenhum vértice $x \in X^*$ ou $y \in Y^*$ coberto por duas arestas este é um emparelhamento em $G$ que cobre todos os vértices de grau máximo. Se não, $E$ tem duas arestas adjacentes. Escolha uma qualquer suponha, s.p.g. que existem duas arestas adjacentes num vértice $x \in X^*$. Obrigatóriamente uma delas pertence a $E_X$, podemos remover esta de $E$ gerando um novo conjunto de arestas que cobre $X^*$ e $Y^*$. Podemos remover arestas até que não exista nenhum par de arestas adjacentes, já que em cada passo mantemos a propriedade de que o conjunto cobre tanto $X^*$ quanto $Y^*$, obtemos um emparelhamento que cobre tanto $X^*$ quanto $Y^*$.
\end{proof}

\begin{ex}
Prove que se $G$ é um grafo $(X,Y)$-bipartido com pelo menos uma aresta e $g(x) \geq g(y)$ para todo $x \in X$ e $y \in Y$, então existe em $G$ um emparelhamento que cobre $X$.
\end{ex}

\begin{proof}
Seja $G$ um grafo $(X,Y)$-bipartido com pelo menos uma aresta e tal que $g(x) \geq g(y)$ para todo par $x \in X$, $y \in Y$. Seja $m$ o valor grau mínimo de um vértice de $X$.

Consideremos o grafo $H$ induzido em $G$ por $S \cup Adj(S)$. Já que $H$ é $(S,Adj(S))$-bipartido, temos que
$$ \sum_{u \in S} g_H(u) = \sum_{v \in Adj(S)} g_H(v) $$, mas
$$ \sum_{u \in S} g_H(u) \geq |S| m $$ e
$$ |Adj(S)| m \geq \sum_{v \in Adj(S)}$$, ou seja
$$ |Adj(S)| m \geq |S| m $$
$$ |Adj(S)| \geq |S| $$
Ou seja, vale o teorema de Hall e existe um emparelhamento que cobre $X$.
\end{proof}

\begin{ex}
Um retângulo latino $m \times n$ é uma matriz com $m$ linhas e $n$ colunas, cujas entradas são símbolos, sendo que cada símbolo ocorre no máximo uma vez em cada linha e em cada coluna. Um quadrado latino de ordem $n$ é um retângulo latino $n \times n$ sobre $n$ símbolos.

Prove: Se $m < n$ então todo retângulo latino $m \times n$ sobre $n$ símbolos pode ser estendido a um quadrado latino de ordem $n$.
\end{ex}

\begin{proof}
Assumindo que os $n$ símbolos possíveis são $s_1, s_2, \dots, s_n$.

Basta montar um grafo $G$ $(X,Y)$-bipartido onde $X = {x_1, x_2, \dots, x_n}$ e $Y = {y_1, y_2, \dots, y_n}$ e para todo par $i, j \in \mathbb{N}$, com $0 \leq i,j \leq n$ tem-se que $x_i$ e $y_j$ são adjacentes entre si se e somente se não existe, na linha $i$, o símbolo $s_j$.

Com isso, temos que, para um $i$ qualquer, o grau de $x_i$ é igual à quantidade de símbolos não utilizados na coluna $i$. Já que $m$ linhas foram usadas e nenhum símbolo se repetiu numa única coluna, $g(x_i) = n-m \forall i$. Por outro lado, para um $j$ qualquer, o grau de $y_i$ é a quantidade de colunas onde o símbolo $s_i$ ainda não foi usado. Sabemos que cada símbolo aparece exatamente uma vez por linha e nunca se repete numa coluna, ou seja, ele já foi usado em exatamente $m$ colunas. Temos então que o $g(y_j) = n-m \forall j$.

Assim, temos um grafo $G$ $(X,Y)$-bipartido onde para todo par $x \in X$ e $y \in Y$, $g(x) \geq g(y)$, pelo exercício anterior, existe um emparelhamento que cobre $X$, já que $|X| = |Y|$, temos um emparelhamento que cobre $X$ e $Y$, assim, se houver uma aresta entre $x_i$ e $y_j$ no emparelhamento para qualquer $i,j$ então podemos colocar o símbolo $s_j$ na coluna $i$ da nova linha e gerar um retângulo latino $m+1 \times n$ para qualquer $m < n$.

Assim, basta repetir o processo acima $n-m$ vezes, a cada passo $n-m > 0$, então o grafo montado vai ter pelo menos uma aresta e ser bipartido, ou seja, vai valer a hipótese do teorema provado acima.
\end{proof}

\end{document}
