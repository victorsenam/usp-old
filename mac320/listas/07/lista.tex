\documentclass[12pt]{article}
\usepackage[utf8]{inputenc}
\usepackage[margin=1in]{geometry} 
%\usepackage[brazil]{babel}
\usepackage{amsmath,amsthm,amssymb,amsfonts}
 
\newcommand{\N}{\mathbb{N}}
\newcommand{\Z}{\mathbb{Z}}
 
\newcounter{exCounter}
\setcounter{exCounter}{22}
\newtheorem{lema}{Lema}
\newtheorem{ex}[exCounter]{Ex}
\newenvironment{problem}[2][Ex]{\begin{trivlist}
\item[\hskip \labelsep {\bfseries #1}\hskip \labelsep {\bfseries #2.}]}{\end{trivlist}}
\linespread{2}
%If you want to title your bold things something different just make another thing exactly like this but replace "problem" with the name of the thing you want, like theorem or lemma or whatever
 
\begin{document}
 
%\renewcommand{\qedsymbol}{\filledbox}
%Good resources for looking up how to do stuff:
%Binary operators: http://www.access2science.com/latex/Binary.html
%General help: http://en.wikibooks.org/wiki/LaTeX/Mathematics
%Or just google stuff
 
\title{Lista 7}
\author{Victor Sena Molero - 8941317}
\maketitle

\section{Exercícios}
\begin{ex}
Seja $G$ um grafo simples de ordem $n \geq 2k$ e tal que $g(v) \geq k \geq 1$ para todo $v$ em $G$. Mostre que $G$ tem um emparelhamento com pelo menos $k$ arestas.
\end{ex}

\begin{proof}[Prova]
Primeiro, vamos provar que se $G$ é um grafo simples de ordem $n \geq 2$ e tal que $g(v) \geq 1$ para todo $v$ em $G$, $G$ tem um emparelhamento com pelo menos 1 aresta. Trivialmente, $G$ tem pelo menos uma aresta pois tem pelo menos um vértice e todo vértice tem grau pelo menos 1, logo, podemos escolher esta aresta qualquer e formar um emparelhamento em $G$ de tamanho 1. Agora, só precisamos resolver o exercício no caso $k > 1$.

Seja $k$ um inteiro positivo maior que  1 e $G$ um grafo simples de ordem $n \geq 2k$ tal que $g(v) \geq k \forall v \in V(G)$. Vamos mostrar que $G$ tem um emparelhamento com pelo menos $k$ arestas.

Se a tese for falsa, teremos um contra-exemplo. Além disso, sabemos que um grafo completo de ordem $n \geq 2k$ tem um emparelhamento de tamanho $k$, logo, temos pelo menos um contra-exemplo diferente do grafo completo. Escolhemos um grafo $G$ não-completo onde valem as hipóteses, não vale a tese e tal que, se adicionarmos uma aresta qualquer ao grafo, vale a tese, ou seja, um contra-exemplo maximal.

Escolhemos então um par qualquer de vértices não adjacentes $u$ e $v$ tal que, se adicionarmos a aresta $uv$ ao grafo $G$ formaremos o grafo $G'$ com um emparelhamento $E'$ de tamanho $k$. Se $uv \notin E'$, então $E'$ é também um emparelhamento em $G$, portanto, $G$ tem um emparelhamento de tamanho $k$.

Caso contrário, escolhemos o emparelhamento $E = E' - uv$, sabemos que $|E| = k-1$ e $u$ e $v$ são não-adjacentes. Seja $X$ o conjunto de vértices adjacentes a $u$ e $Y$ o conjunto de vértices adjacentes a $v$, sabemos que $|X| \geq k$ e $|Y| \geq k$. Suponha, por absurdo que não existe nenhum vértice livre em $X \cup Y$ e não existe alguém em $X$ emparelhado com alguém de $Y$, assim, podemos definir o conjunto $Z$ de vértices emparelhados com vértices de $X$, sabemos que $|X| = |Z|$ e que $Z \cap Y = \emptyset$, então temos que $|Z \cup Y| = |Z| + |Y| = |X| + |Y| \geq 2*k$, porém, só existem $2(k-1)$ vértices emparelhados no grafo, então existe alguém em $Z \cup Y$ não emparelhado, ou seja, livre, um absurdo. Assim, sabemos que existem 3 possibilidades (não disjuntas):

\begin{itemize}
    \item Existe alguém livre em $X$, assim, adicionamos uma aresta de $u$ para tal vértice livre no emparelhamento $E'$ gerando um emparelhamento $E$ de tamanho $k$.
    \item O caso com alguém, livre em $Y$ é análogo.
    \item Existe uma aresta emparelhada $xy$ tal que $x \in X$ e $y \in Y$, assim, existe um caminho alternante $u, x, y, v$ no grafo, que tem duas pontas livres, assim, pode-se gerar um emparelhamento de tamanho $k$.
\end{itemize}

Em todos os casos, então, obtivemos um emparelhamento de tamanho $k$.
\end{proof}

\begin{ex}
Seja $G$ um grafo bipartido com pelo menos uma aresta. Mostre que existe um emparelhamento que cobre todos os vértices de grau $\Delta(G)$
\end{ex}

\begin{proof}
Seja $G$ um grafo $(X,Y)$-bipartido com pelo menos uma aresta. Seja $X^*$ o conjunto de vértices de grau $\Delta(G)$ em $X$. Vamos provar que sempre existe um emparelhamento em $G$ que cobre $X^*$.

Suponha, por absurdo, que existe um subconjunto $S$ de $X^*$ tal que $|Adj(S)| < |S|$. Consideremos o grafo $H$ induzido em $G$ por $S \cup Adj(S)$. Já que $H$ é $(S,Adj(S))$-bipartido, temos que
$$ \sum_{u \in S} g_H(u) = \sum_{v \in Adj(S)} g_H(v) $$, mas
$$ \sum_{u \in S} g_H(u) = |S| \Delta(G) $$ e
$$ \sum_{v \in Adj(S)} \leq |Adj(S)| \Delta(G) < |S| \Delta(G) $$, ou seja
$$ |S| \Delta(G) < |S| \Delta(G) $$, um absurdo.

Temos, então, que para todo subconjunto $S$ de $X^*$, $|Adj(S)| \geq |S|$, portanto, vale o teorema de Hall e existe um emparelhamento em $G$ que cobre $X^*$. 

Analogamente, existe um emparelhamento em $G$ que cobre o conjunto $Y^*$ de vértices de grau $\Delta(G)$ em $Y$.

Escolhemos então um emparelhamento $E_X$ que cobre $X^*$ e um $E_Y$ que cobre $Y^*$. Escolha o conjunto de arestas $E = E_X \cup E_Y$, se não houver nenhum vértice $x \in X^*$ ou $y \in Y^*$ coberto por duas arestas este é um emparelhamento em $G$ que cobre todos os vértices de grau máximo. Se não, $E$ tem duas arestas adjacentes. Escolha uma qualquer suponha, s.p.g. que existem duas arestas adjacentes num vértice $x \in X^*$. Obrigatóriamente uma delas pertence a $E_X$, podemos remover esta de $E$ gerando um novo conjunto de arestas que cobre $X^*$ e $Y^*$. Podemos remover arestas até que não exista nenhum par de arestas adjacentes, já que em cada passo mantemos a propriedade de que o conjunto cobre tanto $X^*$ quanto $Y^*$, obtemos um emparelhamento que cobre tanto $X^*$ quanto $Y^*$.
\end{proof}

\begin{ex}
Prove que se $G$ é um grafo $(X,Y)$-bipartido com pelo menos uma aresta e $g(x) \geq g(y)$ para todo $x \in X$ e $y \in Y$, então existe em $G$ um emparelhamento que cobre $X$.
\end{ex}

\begin{proof}
Seja $G$ um grafo $(X,Y)$-bipartido com pelo menos uma aresta e tal que $g(x) \geq g(y)$ para todo par $x \in X$, $y \in Y$. Seja $m$ o valor grau mínimo de um vértice de $X$.

Consideremos o grafo $H$ induzido em $G$ por $S \cup Adj(S)$. Já que $H$ é $(S,Adj(S))$-bipartido, temos que
$$ \sum_{u \in S} g_H(u) = \sum_{v \in Adj(S)} g_H(v) $$, mas
$$ \sum_{u \in S} g_H(u) \geq |S| m $$ e
$$ |Adj(S)| m \geq \sum_{v \in Adj(S)}$$, ou seja
$$ |Adj(S)| m \geq |S| m $$
$$ |Adj(S)| \geq |S| $$
Ou seja, vale o teorema de Hall e existe um emparelhamento que cobre $X$.
\end{proof}

\begin{ex}
Um retângulo latino $m \times n$ é uma matriz com $m$ linhas e $n$ colunas, cujas entradas são símbolos, sendo que cada símbolo ocorre no máximo uma vez em cada linha e em cada coluna. Um quadrado latino de ordem $n$ é um retângulo latino $n \times n$ sobre $n$ símbolos.

Prove: Se $m < n$ então todo retângulo latino $m \times n$ sobre $n$ símbolos pode ser estendido a um quadrado latino de ordem $n$.
\end{ex}

\begin{proof}
Assumindo que os $n$ símbolos possíveis são $s_1, s_2, \dots, s_n$.

Basta montar um grafo $G$ $(X,Y)$-bipartido onde $X = {x_1, x_2, \dots, x_n}$ e $Y = {y_1, y_2, \dots, y_n}$ e para todo par $i, j \in \mathbb{N}$, com $0 \leq i,j \leq n$ tem-se que $x_i$ e $y_j$ são adjacentes entre si se e somente se não existe, na linha $i$, o símbolo $s_j$.

Com isso, temos que, para um $i$ qualquer, o grau de $x_i$ é igual à quantidade de símbolos não utilizados na coluna $i$. Já que $m$ linhas foram usadas e nenhum símbolo se repetiu numa única coluna, $g(x_i) = n-m \forall i$. Por outro lado, para um $j$ qualquer, o grau de $y_i$ é a quantidade de colunas onde o símbolo $s_i$ ainda não foi usado. Sabemos que cada símbolo aparece exatamente uma vez por linha e nunca se repete numa coluna, ou seja, ele já foi usado em exatamente $m$ colunas. Temos então que o $g(y_j) = n-m \forall j$.

Assim, temos um grafo $G$ $(X,Y)$-bipartido onde para todo par $x \in X$ e $y \in Y$, $g(x) \geq g(y)$, pelo exercício anterior, existe um emparelhamento que cobre $X$, já que $|X| = |Y|$, temos um emparelhamento que cobre $X$ e $Y$, assim, se houver uma aresta entre $x_i$ e $y_j$ no emparelhamento para qualquer $i,j$ então podemos colocar o símbolo $s_j$ na coluna $i$ da nova linha e gerar um retângulo latino $m+1 \times n$ para qualquer $m < n$.

Assim, basta repetir o processo acima $n-m$ vezes, a cada passo $n-m > 0$, então o grafo montado vai ter pelo menos uma aresta e ser bipartido, ou seja, vai valer a hipótese do teorema provado acima.
\end{proof}

\end{document}
