\documentclass[12pt]{article}
\usepackage[utf8]{inputenc}
\usepackage[margin=1in]{geometry} 
\usepackage{amsmath,amsthm,amssymb,amsfonts}
 
\newcommand{\N}{\mathbb{N}}
\newcommand{\Z}{\mathbb{Z}}
 
\newenvironment{problem}[2][Ex]{\begin{trivlist}
\item[\hskip \labelsep {\bfseries #1}\hskip \labelsep {\bfseries #2.}]}{\end{trivlist}}
\linespread{2}
%If you want to title your bold things something different just make another thing exactly like this but replace "problem" with the name of the thing you want, like theorem or lemma or whatever
 
\begin{document}
 
%\renewcommand{\qedsymbol}{\filledbox}
%Good resources for looking up how to do stuff:
%Binary operators: http://www.access2science.com/latex/Binary.html
%General help: http://en.wikibooks.org/wiki/LaTeX/Mathematics
%Or just google stuff
 
\title{Lista 3}
\author{Victor Sena Molero - 8941317}
\maketitle

\begin{problem}{11}
Prove que se G é uma árvore tal que $\Delta(G) \geq k$, então $G$ tem pelo menos $k$ folhas.
\end{problem}

\begin{proof}[Prova]
Seja $G$ uma árvore tal que $\Delta(G) \geq k$ e seja $u$ um vértice de grau máximo em $G$. Definimos $h = g_G(u) = \Delta(G) \geq k$. \\
Se $h = 0$, $k = 0$. Já que toda árvore tem pelo menos uma folha, toda árvore tem pelo menos zero folhas. \\
Se $h > 0$, temos que $u$ tem $h$ vizinhos. Seja $x$ um vizinho de $u$. Vou provar que existe um caminho $S_u$ de $u$ para uma folha $x'$ qualquer que tem como primeira aresta a $\{u, x\}$. Para encontrar este caminho basta pegar o maior caminho que começa em $S_x$ e tem como primeira aresta $\{u, x\}$, agora, se o caminho não termina em uma folha, é possível extendê-lo, pois o ultimo vértice tem mais de uma aresta e só uma delas pertence ao caminho (já que ele é uma folha), portanto, só um dos vértices vizinhos pertencem ao caminho, se não, seria possível achar um ciclo em $G$. Ou seja, o maior caminho que começa em $u$ e tem como primeira aresta $\{u, x\}$ termina em uma folha. \\
Agora escolhemos dois vértices $x$ e $y$ vizinhos de $u$, vamos provar que se os caminhos $S_x$ e $S_y$ têm mais de um vértice em comum, então $x = y$. Basta assumir que $S_x$ e $S_y$ têm mais de um vértice em comum e, agora, assumir, por absurdo que $x$ e $y$ são distintos. $S_x$ e $S_y$ começam em $u$ e tem mais um vértice $z$ em comum. Sabemos que a primeira aresta de $S_x$ é $\{u,x\}$ e a primeira de $S_y$ é $\{u,y\}$. Logo, $z$ não vem está entre $u$ e $x$ no caminho $S_x$ e nem entre $u$ e $y$ no caminho $S_y$. Assim, existe um caminho de $u$ para $z$ que tem como primeira aresta $\{u,x\}$ e um caminho de $u$ para $z$ que tem como primeira aresta $\{u,y\}$, logo, existem dois caminhos distintos de $u$ para $z$, um absurdo. \\
Logo, cada vizinho de $u$ gera um caminho para uma folha distinta. Já que existem $h$ vizinhos de $u$, encontramos $h$ folhas distintas em $G$. Então, $G$ tem pelo menos $h$ folhas, e $h \geq k$, logo, $G$ tem pelo menos $k$ folhas.
\end{proof}

\begin{problem}{12}
Prove que um grafo conexo $G$ possui pelo menos $|A(G)| - |V(G)| + 1$ circuitos.
\end{problem}

\begin{proof}[Prova]
Seja $G$ um grafo conexo com $|V(G)|$ vértices. E seja $|C(G)|$ a quantidade de ciclos de $G$.\\
Se $|A(G)| = |V(G)| - 1$, então $G$ é uma árvore. Logo, $|C(G)| = 0$, ou seja, $|C(G)| \geq |A(G)| - |V(G)| + 1 = 0$. \\
Assuma que, com $|A(G)| = k \geq |V(G)| - 1$, $|C(G)| \geq |A(G)| - |V(G)| + 1$.
Se $|A(G)| = k + 1$, então, $G$ é conexo e não é uma árvore. Logo, existe pelo menos uma aresta de $G$ que não é uma ponte. Se removermos uma aresta $a$ qualquer de $G$ que não seja uma ponte, geramos o grafo $G'$. Já que $a$ não é uma ponte em $G$, existe um ciclo em $G$ que passa por $a$ e, já que $a$ não está em $G'$ este ciclo não existe em $G'$, além disso, $G' \subseteq G$, logo, todo ciclo de $G'$ é também um ciclo de $G$, portanto, $|C(G')| < |C(G)|$. \\
Além disso, $G'$ é um grafo conexo com $k$ arestas, portanto, vale a hipótese indutiva e $|C(G')| \geq |A(G')| - |V(G')| + 1 = |A(G)| - |V(G)|$. Ou seja $|C(G)| > |C(G')| \geq |A(G)| - |V(G)|$, logo, $|C(G)| \geq |A(G)| - |V(G)| + 1$. \\
Ou seja, por indução, $\forall G$ conexo, $|C(G)| \geq |A(G)| - |V(G)| + 1$.
\end{proof}

\begin{problem}{13}
Existem grafos simples com exatamente duas árvores geradoras distintas? Justifique.
\end{problem}

\begin{proof}[Resposta]
Não existem. \\
Seja $G$ um grafo simples. Vamos separar em três casos. \\
Se $|A(G)| < |V(G)| - 1$ o grafo é desconexo e não possui nenhuma árvore geradora. \\
Se $|A(G)| = |V(G)| - 1$ o grafo é uma árvore, logo, possui apenas uma árvore geradora. \\
Se $|A(G)| > |V(G)| - 1$ o grafo possúi pelo menos um ciclo. Vou provar, por indução, que é possível remover arestas de $G$ mantendo ele conexo até que ele tenha exatamente $|V(G)|$. \\
Se $|A(G)| = |V(G)|$, então basta não remover nenhuma aresta. \\
Assumindo que vale se $|A(G)| = k \geq |V(G)|$, se $|A(G)| = k + 1$, $G$ tem pelo menos um ciclo, logo, tem pelo menos uma aresta que não é ponte, basta remover ela, o grafo continuará conexo e terá $k$ arestas, vale a hipótese de indução, logo é possível continuar removendo arestas até que $|A(G)| = |V(G)|$. \\
Agora, aplicamos o que acabamos de provar para obter um grafo $G'$ com os mesmos vértices de $G$, mas com exatamente $|V(G)|$ arestas tais que $A(G') \subseteq A(G)$. Uma árvore geradora de $G'$ é uma árvore geradora de $G$, pois é um conjunto de arestas que pertencem a $G$, tornam $G$ conexo e formam uma árvore. \\
Porém, $G'$ tem um ciclo, já que todo ciclo tem pelo menos 3 arestas, existem pelo menos 3 arestas que podemos remover de $G'$ para criar uma árvore, logo, $G'$ tem pelo menos 3 árvores geradoras, portanto $G$ tem pelo menos 3 árvores geradoras. \\
Assim, se conclui que nenhum grafo $G$ simples tem exatamente 2 árvores geradoras.
\end{proof}

\begin{problem}{14}
Prove que todo grafo conexo $G$, simples e não-trivial, tem uma árvore geradora $T$ tal que $G - A(T)$ é desconexo.
\end{problem}

\begin{proof}[Prova]
Vou primeiro provar que, seja $G$ um grafo conexo, simples e não-trivial, se existe um particionamento em vértices de $G$ com uma partição $H$ tal que, para toda outra partição $P$ existe uma aresta entre $H$ e $P$ e tal que cada partição possui uma árvore geradora, então existe uma árvore geradora de $G$ que contém as árvores geradoras de cada uma das partições. Vou provar com indução na quantidade de partições. É fácil ver que existe pelo menos uma partição, já que é necessário que exista a partição $H$ em questão. Logo, seja $n$ é a quantidade de partições: \\
Se $n = 1$ temos apenas uma partição, logo, ela possui todos os vértices de $G$ e já ela tem uma árvore geradora, $G$ tem uma árvore geradora que contém as árvores geradoras de todas as partições. \\
Suponha que com $n = k \geq 1$ a afirmação seja verdadeira. Se $n = k+1$, podemos escolher uma partição $P$ qualquer do grafo que não seja $H$ e remover todos vértices dela do grafo, gerando assim um grafo $G'$ com um particionamento em vértices de $k$ partições tal que cada uma delas possui uma árvore geradora e existe uma partição $H$ tal que exista, em $G'$, uma aresta entre $H$ e cada uma das partições restantes. Vale, então, a hipótese de indução e $G'$ tem uma árvore geradora que contém as árvores geradoras de cada uma das suas partições. Agora, basta adicionar a esta árvore a árvore geradora e $P$ e uma aresta entre $P$ e $H$ para formar uma árvore geradora de $G$ que contém as árvores de cada uma das partições. \\
Agora, vamos à prova principal. Seja $G$ um grafo conexo, simples e não-trivial. Seja $u$ um vértice qualquer de $G$. Se $G$ é não trivial, $g_G(u) > 0$. Assim, $u$ tem pelo menos um vizinho. \\
Podemos escolher o subgrafo induzido de $H$ de $G$ que contém somente $u$ e todos os vizinhos de $u$. O conjunto de todas as arestas de $u$ forma uma árvore geradora $T_H$ de $H$. \\
Seja $G' = G \setminus H$, $G'$ não é necessáriamente conexo. Cada uma das componentes de $G'$ possui uma árvore geradora. Cada uma das componentes $K$ de $G'$ possui uma árvore geradora e existe, em $G$, pelo menos uma aresta que leva de $H$ em $K$. Ou seja, temos um particionamento em vértices de $G$ onde cada partição tem uma árvore geradora e onde temos uma partição $H$ tal que existe, em $G$, pelo menos uma aresta entre cada um das outras partições e $H$. \\
Ou seja, como provamos anteriormente, $G$ tem uma árvore geradora $T$ que contém a árvore geradora $T_H$ de $H$. Já que $T_H$ contém todas as arestas de $u$, remover as arestas dela desconecta o grafo, pois isola o vértice $u$. Então remover as arestas de $T$ de $G$ desconecta $G$, logo, $G - A(T)$ é desconexo. Então, qualquer grafo $G$ conexo, simples e não-trivial tem uma árvore geradora $T$ tal que $G - A(T)$ é desconexo.
\end{proof}

\end{document}
