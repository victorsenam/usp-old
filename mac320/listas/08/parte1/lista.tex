\documentclass[12pt]{article}
\usepackage[utf8]{inputenc}
\usepackage[margin=1in]{geometry} 
%\usepackage[brazil]{babel}
\usepackage{amsmath,amsthm,amssymb,amsfonts,enumitem}
 
\newcommand{\N}{\mathbb{N}}
\newcommand{\Z}{\mathbb{Z}}
 
\newcounter{exCounter}
\setcounter{exCounter}{26}
\newtheorem{lema}{Lema}
\newtheorem{ex}[exCounter]{Ex}
\newenvironment{problem}[2][Ex]{\begin{trivlist}
\item[\hskip \labelsep {\bfseries #1}\hskip \labelsep {\bfseries #2.}]}{\end{trivlist}}
\linespread{2}
%If you want to title your bold things something different just make another thing exactly like this but replace "problem" with the name of the thing you want, like theorem or lemma or whatever
 
\begin{document}
 
%\renewcommand{\qedsymbol}{\filledbox}
%Good resources for looking up how to do stuff:
%Binary operators: http://www.access2science.com/latex/Binary.html
%General help: http://en.wikibooks.org/wiki/LaTeX/Mathematics
%Or just google stuff
 
\title{Lista 8}
\author{Victor Sena Molero - 8941317}
\maketitle

\section{Exercícios}
\begin{ex}
Seja $G$ um grafo de ordem $n$. Mostre que se $n$ é ímpar e $G$ tem mais do que $\Delta(G)(n-1)/2$ arestas, então $\chi'(G) > \Delta(G)$.
\end{ex}

\begin{proof}[Resposta]
Não sei :(
\end{proof}

\begin{ex}
\begin{enumerate}[label=(\alph*)]
    \item Mostre que se $G$ é um grafo bipartido, então $G$ tem um supergrafo bipartido $k$-regular, onde $k = \Delta(G)$.
    \item Usando o resultado do item $(a)$ faça uma prova alternativa do Teorema de König (Teorema 6.2 das Notas de Aula).
\end{enumerate}
\end{ex}

\begin{proof}[Prova da parte a]
Primeiro, seja $G$ um grafo $(X,Y)$-bipartido, podemos assumir s.p.g. que $|X| \geq |Y|$. Podemos escolher um supergrafo $G'$ de $G$ que é $(X,Y')$-bipartido onde $Y'$ contém $Y$ e $|X|-|Y|$ vértices de grau 0. \\
Temos agora que se $G'$ tem um grafo $\Delta(G')$-regular, temos que $G$ tem um supergrafo $\Delta(G')$-regular e já que $\Delta(G) = \Delta(G')$, $G$ tem um supergrafo $\Delta(G)$-regular.

Escolhemos agora o grafo $G'$ e vamos provar, por indução na quantidade de arestas de $G'$, que $G'$ tem um supergrafo $\Delta(G')$-regular. 

Se $|A(G')| = |X|\Delta(G')$, então $G'$ é $\Delta(G')$-regular, trivialmente.

Caso contrário, podemos assumir, por hipótese de indução, que todo grafo que cumpre as hipóteses e tem mais do que $|A(G')|$ arestas segue a tese. Sabemos que existe um vértice de $X$ que não tem grau $\Delta(G')$, mas sabemos também que
$$ \sum\limits_{u \in X} g_{G'}(u) = \sum\limits_{v \in Y} g_{G'}(v) < |X|\Delta(G') = |Y|\Delta(G') $$
Então tem que haver um vértice em $Y$ que não tem grau $\Delta(G')$ também, logo, podemos escolher estes dois vértices e adicionar uma aresta entre eles. Geramos assim um supergrafo de $G'$ com $|A(G')|+1$ que, é $(X,Y')$-bipartido e tem grau máximo $\Delta(G')$. Por hipótese de indução, este tem um supergrafo bipartido $\Delta(G')$-regular, logo, $G'$ tem um supergrafo bipartido $\Delta(G')$-regular.

Assim, $G$ tem um supergrafo bipartido $\Delta(G)$-regular.
\end{proof}

\begin{proof}[Prova da parte b]
O Teorema de König diz que se $G$ é um grafo bipartido, então $\chi'(G) \leq \Delta(G)$. Então seja $G$ um grafo bipartido qualquer, podemos usar o teorema do item $(a)$ para concluir que que existe um supergrafo $H$ de $G$ que é $\Delta(G)$-regular. Vamos provar que se $H$ é um grafo $k$-regular então ele é $k$ colorível, por indução em $k$.

Se $k = 0$ então é não há nenhuma aresta e é possível colorir ele com nenhuma cor, trivialmente. 

Se $k > 0$, assumimos que a tese é válida para todo $k' = k$. Sabemos que todo grafo bipartido regular tem um emparelhamento perfeito, assim, removemos este emparelhamento de $H$ obtendo $H'$ $k-1$-regular, pela hipótese de indução, $H'$ é $k-1$ colorível, podemos escolher o emparelhamento de $H$ para gerar a nova cor em $H$ e conseguir uma $k$ coloração em $H$.

Assim, provamos que $H$ é $\Delta(G)$-colorível, basta tirar de cada cor as arestas que não estão em $G$ e obter uma $\Delta(G)$ coloração em $G$, ou seja $\chi'(G) \leq \Delta(G)$.
\end{proof}

\begin{ex}
Seja $G$ um grafo que tem uma coloração própria na qual toda cor é usada pelo menos 2 vezes. Mostre que $G$ tem uma coloração com $\chi(G)$ cores que tem essa mesma propriedade.
\end{ex}

\begin{proof}[Resposta]
Não sei :(
%Seja $G$ um grafo simples tal que toda coloração minima tem pelo menos uma cor usada só uma vez. Escolhemos uma coloração $C$ de $G$ minima que minimize a quantidade de cores com só um vértice.

%Seja $c$ uma cor de $C$ que só tem um vértice $u$. Vamos provar que em toda coloração de $G$ existe algum vértice que aparece sozinho.

%Seja $d$ uma outra cor qualquer de $C$. Se $d$ só tem um vértice, ele é adjacente a $u$, se não poderíamos formar uma coloração menor em $G$. Se $d$ tem mais do que dois vértices, todos são adjacentes a $u$, se não, poderíamos escolher um qualquer e dar a ele a cor $c$, gerando uma coloração com menos cores de um só vértice. Se $d$ tem mais de um vértice, então pelo menos um deles é adjacente a $u$, se não poderíamos atribuir a cor $d$ também a $u$. 

%Escolhemos então uma coloração $D$ do grafo onde $u$ não aparece sozinho numa cor. Temos que
\end{proof}

\begin{ex}
Seja $G$ um grafo simples com $n$ vértices e seja $\alpha$ a cardinalidade de um conjunto independente máximo de $G$. Prove que
\begin{enumerate}[label=(\alph*)]
    \item $n/\alpha \leq \chi(G) \leq n - \alpha + 1$
    \item Caracterize os grafos $G$ de ordem $n$ tais que $\chi(G) = n - \alpha + 1$.
\end{enumerate}
\end{ex}

\begin{proof}[Prova da parte a]
Seja $G$ um grafo simples com $n$ vértices onde $\alpha$ é a cardinalidade máxima de um conjunto independente. Seja também $C$ uma coloração qualquer desse grafo, temos que
$$ n = \sum\limits_{c \in C} |c| \leq \sum\limits_{c \in C} \alpha = |C| \alpha $$, então
$$ n \leq |C| \alpha $$, logo
$$ n/\alpha \leq |C| $$, ou seja
$$ n/\alpha \leq \chi(G) $$
Por outro lado temos que, já que existe um tamanho independente de tamanho $\alpha$, podemos colorir ele todo de uma cor, restam-nos $n-\alpha$ vértices. Se escolhermos uma cor distinta para cada um, teremos uma coloração de $n-\alpha+1$ cores, logo $\chi(G) \leq n-\alpha+1$.
\end{proof}

\begin{proof}[Resposta da parte b]
Se $G$ é tal que $\chi(G) = n - \alpha + 1$ temos que a coloração feita da forma descrita no exercício anterior é mínima. Assim, podemos concluir que todos os vértices que não estão no conjunto independente de tamanho $\alpha$ são adjacentes a todos os outros que não pertencem ao conjunto. Além disso, cada um desses vértices é adjacente a pelo menos um vértice do conjunto independente, caso contrário, haveria um conjunto independente de tamanho $\alpha + 1$. Ou seja, este grafo tem um grafo completo de tamanho $n - \alpha$ e $\alpha$ vértices não-adjacentes entre si que são adjacentes ao grafo completo. 
\end{proof}

\end{document}
