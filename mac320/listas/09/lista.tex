\documentclass[12pt]{article}
\usepackage[utf8]{inputenc}
\usepackage[margin=1in]{geometry} 
\usepackage[brazil]{babel}
\usepackage{amsmath,amsthm,amssymb,amsfonts,enumitem,tikz}
 
\newcommand{\N}{\mathbb{N}}
\newcommand{\Z}{\mathbb{Z}}
 
\newcounter{exCounter}
\setcounter{exCounter}{31}
\newtheorem{lema}{Lema}
\newtheorem{ex}[exCounter]{Ex}
\newenvironment{problem}[2][Ex]{\begin{trivlist}
\item[\hskip \labelsep {\bfseries #1}\hskip \labelsep {\bfseries #2.}]}{\end{trivlist}}
\linespread{2}
%If you want to title your bold things something different just make another thing exactly like this but replace "problem" with the name of the thing you want, like theorem or lemma or whatever
 
\begin{document}
 
%\renewcommand{\qedsymbol}{\filledbox}
%Good resources for looking up how to do stuff:
%Binary operators: http://www.access2science.com/latex/Binary.html
%General help: http://en.wikibooks.org/wiki/LaTeX/Mathematics
%Or just google stuff
 
\title{Lista 9}
\author{Victor Sena Molero - 8941317}
\maketitle

\section{Exercícios}
\begin{ex}[a]
Exiba um grafo planar de ordem 8 cujo complemento também é planar.
\end{ex}
\setcounter{exCounter}{31}

\begin{proof}[Resposta]
Queremos criar um grafo planar $G$ de grau 8 cujo complemento também é planar. Definimos $H_{a,b}$ para qualquer par de inteiros $a,b$ onde $a \leq b$ como um grafo completo com vértices $\{a,a+1,\dots,b\}$. Agora, basta definir $V(G) = \{1,2,\dots,8\}$ e $A(G) = A(H_{1,4}) \cup A(H_{3,6}) \cup A(H_{5,8})$. Tanto $G$ quanto seu complementar são planares.
\end{proof}

\begin{ex}[b]
Exiba um grafo não planar cujo complemento não é planar.
\end{ex}

\begin{proof}[Resposta]
Para obter tal grafo, basta escolher um $K_5$ e adicionar 5 novos vértices de grau 1, cada um conectado a uma aresta distinta do $K_5$. Este grafo é não planar pois contém um $K_5$ e, em seu complemento, todos os vértices novos são adjacentes entre si, formando, também, um $K_5$.
\end{proof}

\begin{ex}
Mostre que se $G$ é um grafo simples conexo planar com cintura $k \geq 3$, então
$$ |A(G)| \leq k(|V(G)| - 2)/(k-2) $$
\end{ex}

\begin{proof}[Prova]
Foi mostrado na apostila, sob o nome de corolário 8.6, que Se $G$ é um grafo simples com $n \geq 3$ vértices e $m$ arestas, então $m \leq 3n - 6$. Queremos mostrar que se $G$ é um grafo simples conexo planar com cintura $k \geq 3$, então $ |A(G)| \leq k(|V(G)| - 2)/(k-2) $. Já que $G$ tem cintura pelo menos 3, então $|V(G)| \geq 3$, logo, ele vale na hipótese do corolário citado, assim, $|A(G)| \leq 3|V(G)| - 6 = 3(|V(G)| - 2)$. Já que $3 \leq k/(k-2) \forall k \geq 3$, então $|A(G)| \leq k(|V(G)| - 2)/(k-2)$.
\end{proof}

\begin{ex}
Mostre que se $|V(G)| = 11$ então $G$ ou seu complemento não é planar.
\end{ex}

\begin{proof}[Prova]
Seja $G$ um grafo tal que $|V(G)| = 11$, suponha, por absurdo que $G$ e $\bar{G}$ são planares. Definimos $a = max(|V(G)|, |V(\bar{G})|$ e $b = min(|V(G)|, |V(\bar{G})|)$, assim $a + b = 11$ e, já que $G$ e seu complementar são planares, $a \leq 3|V(G)| - 6 = 27$ e $b \leq 27$, mas
$$ a+b = 55 $$
$$ 55-b = a \leq 27 $$
$$ b \geq 28 $$
Um absurdo, logo, ou $G$ ou seu complementar são não-planares.
\end{proof}

\begin{ex}
Um grafo planar $G$ é \textit{auto-dual} se é isomorfo ao seu dual (geométrico) $G^*$.
\end{ex}

\setcounter{exCounter}{34}
\begin{ex}[a]
Mostre que se $G$ é auto-dual, então $2|V(G)| = |A(G)| + 2$.
\end{ex}

\begin{proof}[Prova]
Seja $G$ um grafo planar auto-dual e $G^*$ seu dual. Já que $G$ é isomorfo a $G^*$, $|V(G)| = |V(G^*)|$. Já que $G$ é planar $|V(G)| - |A(G)| + |F(G)| = 2$. Além disso, se $G$ e $G^*$ são duais, $|F(G)| = |V(G^*)|$. Assim
$$ |F(G)| = |V(G^*)| = |V(G)| $$, assim
$$ |V(G)| - |A(G)| + |V(G)| = 2 $$, ou seja
$$ 2|V(G)| = |A(V)| + 2 $$
\end{proof}

\setcounter{exCounter}{34}
\begin{ex}[b]
Mostre que nem todo grafo $G$ com $2|V(G)| = |A(G)| + 2$ é auto-dual.
\end{ex}

\begin{proof}[Prova]
Basta escolher um triângulo e adicionar um loop em um dos vértices. Este é um grafo que respeita a igualdade dada e não é auto-dual, seu dual é um grafo de três vértices, onde um par deles é conectado por 3 arestas, outro par é conectado por uma aresta e o par restante não é adjacente.
\end{proof}


\end{document}
