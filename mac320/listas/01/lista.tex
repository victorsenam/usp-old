\documentclass[12pt]{article}
\usepackage[utf8]{inputenc}
\usepackage[margin=1in]{geometry} 
\usepackage{amsmath,amsthm,amssymb,amsfonts}
 
\newcommand{\N}{\mathbb{N}}
\newcommand{\Z}{\mathbb{Z}}
 
\newenvironment{problem}[2][Ex]{\begin{trivlist}
\item[\hskip \labelsep {\bfseries #1}\hskip \labelsep {\bfseries #2.}]}{\end{trivlist}}
\linespread{2}
%If you want to title your bold things something different just make another thing exactly like this but replace "problem" with the name of the thing you want, like theorem or lemma or whatever
 
\begin{document}
 
%\renewcommand{\qedsymbol}{\filledbox}
%Good resources for looking up how to do stuff:
%Binary operators: http://www.access2science.com/latex/Binary.html
%General help: http://en.wikibooks.org/wiki/LaTeX/Mathematics
%Or just google stuff
 
\title{Lista 1}
\author{Victor Sena Molero - 8941317}
\maketitle

\begin{problem}{E1}
Nos dois items deste problema vamos assumir que o grafo $G$ é bipartido em $G_1$ e $G_2$. Também vamos assumir s.p.g. que $|V(G_1)| \leq |V(G_2)|$ e definir $n = |V(G)|$ e $a = |V(G_1)|$. É fácil ver que $|V(G_2)| = n - a$. Sabemos também que o máximo de arestas que este grafo pode ter é $a(n-a)$. \\
\end{problem}

\begin{problem}{E1 (a)}
Prove que um grafo simples de ordem $n$ com mais do que $n^2/4$ arestas não é bipartido.
\end{problem}

\begin{proof}
Sabemos que $a \leq n/2$, já que $|V(G_1)| \leq |V(G_2)|$. \\
Se $a < n/2$, existe um $k$ tal que $n/2 \geq k > 0$ e $a = n/2-k$. Temos que $|A(G)| \leq a(n-a) = (n/2-k)(n-n/2+k) = n^2/4 - k^2 < n^2/4$.
E, também, se $a = n/2$, temos que $|A(G)| \leq a(n-a) = (n/2)^2 = n^2/4$. \\
Logo, é impossível que $|A(G)| > n^2/4$.
\end{proof}

\begin{problem}{E1 (b)}
Encontre todos (diga como são) os grafos bipartidos de ordem $n$ com $\lfloor n^2/4 \rfloor$ arestas. Justifique. \\
\end{problem}

\begin{proof}
Estes grafos tem lados de grau $\lfloor n/2 \rfloor$ e $\lceil n/2 \rceil$ e são completos. \\
Sabemos que $a \leq \lfloor n/2 \rfloor$, já que $|V(G_1)| \leq |V(G_2)|$. \\
Se $n$ é ímpar e $a < \lfloor n/2 \rfloor$, temos que existe um $k$ tal que $\lfloor n/2 \rfloor \geq k > 0$ e $a = \lfloor n/2 \rfloor -k$. Temos, também que $\lfloor n/2 \rfloor = (n-1)/2$. Assim:
$$ a(n-a) = ((n-1)/2 - k)(n - (n-1)/2 + k) = ((n-1)/2 - k)((n+1)/2 + k) = $$
$$ = (n/2 - (1+2k)/2)(n/2 + (1+2k)/2) = n^2/4 - ((1+2k)/2)^2 = n^2/4 - (k+1)^2 \leq $$
$$ \leq n^2/4 - 1 < \lfloor n^2/4 \rfloor $$ \\
Se $n$ é par, temos que $\lfloor n/2 \rfloor = n/2$ e $\lfloor n^2/4 \rfloor$ e já provamos no item a que $a < n/2 \implies a(n-a) < n^2/4 = \lfloor n^2/4 \rfloor$. \\
Agora, se $n = \lfloor n/2 \rfloor$ com um $n$ par, temos que $a(n-a) = n^2/4 = \lfloor n^2/4 \rfloor$. Mas com um $n$ ímpar temos $a(n-a) = (n-1)/2 (n+1)/2 = (n^2-1)/4$, por outro lado, $n^2$ é impar e
$$ \lfloor n^2/4 \rfloor = (n^2 - 1)/4 \quad \text{ ou } \quad \lfloor n^2/4 \rfloor = (n^2-3)/4 $$ \\
já que $n^2/4 - \lfloor n^2/4 \rfloor < 1$. E sabemos que $(n^2-1)/4 = (n-1)/2 (n+1)/2$, logo, é inteiro, por ser produto de dois inteiros, ou seja, $(n^2-3)/4$ não é inteiro. Portanto, $ \lfloor n^2/4 \rfloor = (n^2-1)/4 $. \\
Ou seja, os únicos grafos bipartidos de ordem $n$ que conseguem ter $\lfloor n^2/4 \rfloor$ arestas tem $a = \lfloor n/2 \rfloor$. E, já que, como vimos acima, $\lfloor n/2 \rfloor \lceil n/2 \rceil = \lfloor n^2/4 \rfloor$ estes grafos devem ser completos.
\end{proof}
 
\begin{problem}{E2}
Existe um grafo bipartido simples $G$ tal que $\delta(G) + \Delta(G) = |V(G)|$? Justifique.
\end{problem}

\begin{proof}
Não. \\
Assuma que o grafo $G$ é bipartido em $G_1$ e $G_2$. Assumimos também, s.p.g., que $|V(G_1)| \leq |V(G_2)|$. \\
Assim, o grau máximo possível para um nó no grafo é $|V(G_2)|$ e o grau máximo possível para um nó em $G_2$ é $|V(G_1)|$, ou seja, $\Delta(G) \leq |V(G_2)| \text{ e } \delta(G) \leq |V(G_1)|$, logo:
$$ \delta(G) + \Delta(G) \leq |V(G_1)| + |V(G_2)| = |V(G)| $$
\end{proof}

\begin{problem}{E3}
Um grafo simples é auto-complementar se é isomorfo ao seu complemento. É possível que um grafo auto-complementar de ordem 100 tenha exatamente um vértice de grau 50? Justifique.
\end{problem}
 
\begin{proof}
O grafo $G$ tem ordem $100$ e um vértice $u$ de grau $50$. Não existe nenhum outro vértice em $G$ com grau $50$. Existe um grafo $\bar{G}$ que é complementar a $G$. \\
Já que $\bar{G}$ é complementar a $G$, existe um vértice $v \in \bar{G}$ tal que $g_{\bar{G}}(v) = 49$. Além disso, não existe nenhum outro vértice $v' \in \bar{G}$ tal que $g_{\bar{G}}(v) = 49$ pois, se houvesse, haveria um outro vértice $u' \in G$ tal que $g_{G}(u) = 50$. \\
Agora, vamos assumir, por absurdo, que $G \cong \bar{G}$. Temos que existe um, e somente um, vértice $u$ de grau $50$ em $G$ (e em $\bar{G}$) e, também, que existe um, e somente um, vértice $v$ de grau $49$ em $G$ (e em $\bar{G}$). \\
Temos dois casos: $u$ é adjacente a $v$ em $G$ ou não. \\
No primeiro caso, $G$ tem seu único vértice de grau $50$ adjacente ao seu único vértice de grau $49$ e $\bar{G}$ não tem nenhum vértice de grau $50$ adjacente a um vértice de grau $49$, logo, eles não são isomorfos, um absurdo. \\
No segundo caso, $G$ tem não tem nenhum vértice de grau $50$ adjacente a um vértice de grau $49$, enquanto $\bar{G}$ tem uma aresta entre um vértice de grau $50$ e um vértice de grau $49$. Assim, eles não são isomorfos, um absurdo. \\
Já que não existe nenhum caso onde não atingimos um absurdo, a hipótese inicial é falsa e podemos afirmar que não é possível que um grafo auto-complementar de ordem 100 tenha exatamente um vértice de grau 50.
\end{proof}

\begin{problem}{E4}
Prove se um grafo $G$ tem exatamente dois vértices de grau ímpar, então $G$ tem um caminho cuja origem e cujo término são precisamente esses vértices.
\end{problem}

\begin{proof}
Por definição, em uma componente conexa, existe um caminho entre cada par de vértices. \\
Vamos provar que, se um grafo $G$ contém exatemente 2 vértices de grau ímpar, ambos pertencem à mesma componente. \\
Primeiro, assumimos, por absurdo que existe uma componente de $G$ com um só vértice $v$ de grau ímpar. Esta componente forma um subgrafo induzido $G'$ onde $g_{G'}(v) = g_G(v) \quad \forall v \in G'$. \\
E sabemos que, para todo grafo $H$, vale $\sum_{v \in H} g_H(v) = 2|V(H)|$, logo
$$ \sum_{v \in G'} g_{G'}(v) = 2|V(G')| $$, \\
porém, o primeiro lado da equação soma apenas um valor ímpar com outros pares, logo é ímpar, e o lado direito é o produto de 2 com um natural, portanto é par. Ou seja, a equação não vale em $G'$, um absurdo. \\
Assim, é impossível que uma componente conexa contenha exatamente um vértice de grau ímpar, logo, os dois devem estar na mesma componente, ou seja, há um caminho entre os dois.
\end{proof}

\begin{problem}{E5}
Seja $G$ um grafo simples. É possível que $G$ e $\bar{G}$ sejam desconexos? Justifique.
\end{problem}

\begin{proof}
Vamos provar que, para todo grafo $G$ de grau $n \geq 1$, $G$ é conexo ou $\bar{G}$ é conexo. \\
Se $n = 1$, $G$ é conexo. \\
Vamos assumir que para um $n \geq 1$, $G$ é conexo ou $\bar{G}$ é conexo e provar que para $n+1$, o mesmo vale. \\
Temos um grafo $H$ de grau $n+1$, removemos um vértice arbritário $v$ de $H$ formando um grafo $G$ de grau $n$. Pela hipótese de indução, $G$ é conexo ou $\bar{G}$ é conexo. Vamos assumir s.p.g. que $G$ é conexo. \\
Assim, se $g_H(v) \neq 0$, $H$ é conexo, pois vai existir um caminho entre $v$ e pelo menos um vértice de $G$ e, já que existe um caminho entre todo par de vértices de $G$, existe um caminho entre $v$ e qualquer vértice de $G$. Logo, existe um caminho entre todo par de vértices de $H$ e, então, $H$ é conexo. \\
Por outro lado, se $g_H(v) = 0$, $\bar{H}$ é conexo, pois, $v$ será adjacente a todo vértice de $\bar{G}$ em $\bar{H}$, ou seja, vai haver um caminho entre todo par de vértices de $\bar{H}$. \\
Logo, $H$ é conexo ou $\bar{H}$ é conexo e, portanto, vale a indução.
\end{proof}

\end{document}
