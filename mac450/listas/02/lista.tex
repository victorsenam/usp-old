\documentclass[12pt]{article}
\usepackage[utf8]{inputenc}
\usepackage[margin=1in]{geometry} 
\usepackage[brazil]{babel}
\usepackage{amsmath,amsthm,amssymb,amsfonts,enumitem,tikz}
 
\newcommand{\N}{\mathbb{N}}
\newcommand{\Z}{\mathbb{Z}}
 
\newcounter{exCounter}
\setcounter{exCounter}{2}
\newtheorem{lema}{Lema}
\newtheorem{ex}[exCounter]{Ex}
\newenvironment{problem}[2][Ex]{\begin{trivlist}
\item[\hskip \labelsep {\bfseries #1}\hskip \labelsep {\bfseries #2.}]}{\end{trivlist}}
\linespread{2}
%If you want to title your bold things something different just make another thing exactly like this but replace "problem" with the name of the thing you want, like theorem or lemma or whatever
 
\begin{document}
 
%\renewcommand{\qedsymbol}{\filledbox}
%Good resources for looking up how to do stuff:
%Binary operators: http://www.access2science.com/latex/Binary.html
%General help: http://en.wikibooks.org/wiki/LaTeX/Mathematics
%Or just google stuff
 
\title{Lista 2}
\author{Victor Sena Molero - 8941317}
\maketitle

\section{Exercícios}
\begin{ex}
Construa instâncias do \textsc{MinCC} com custos unitários, ou seja, instâncias $(E,\mathcal{S},c)$ com $c_S = 1$ para todo $S$ em $\mathcal{S}$, para as quais o custo da cobertura produzida pelo algoritmo $\textsc{MinCC-Chvátal}$ pode chegar arbitrariamente perto de $H_n\mathrm{opt}(E,\mathcal{S},c)$, onde $n := |E|$.
\end{ex}

\begin{proof}[Resposta]
Vamos construir uma instância $I = (E,\mathcal{S},c)$ que alcança a aproximação pedida no algoritmo $\textsc{MinCC-Chvátal}$. Seja $n$ um inteiro positivo, construiremos $E$ tal que $|E| = n*n$ e então podemos indexar os elementos de $E$ em uma matriz $n \times n$, ou seja, identificar cada um dos elementos de $E$ por um par $(i,j)$ e denotar o elemento em questão por $E_{i,j}$.  

Precisamos agora descrever os conjuntos contidos em $\mathcal{S}$. Teremos $n$ conjuntos que contém, cada um, uma coluna distinta da matriz, ou seja, para todo $i \in [1, n]$ existe exatamente um $S^*_i \in \mathcal{S}$ tal que $S_i = \{E_{j,i} \mid j \in [1, n]\}$. Denotaremos o conjunto de todos os $S^*_i$ por $\mathcal{S}^*$.

Além disso, teremos vários outros conjuntos em $\mathcal{S}$ que particionam cada uma das linhas da matriz $E$ separadamente. Cada linha será particionada em 1 ou mais conjuntos de mesmo tamanho. Mais especificamente, a $i$-ésima linha será dividida em $\lfloor \frac{n}{i} \rfloor$ conjuntos de tamanho $n/\lfloor \frac{n}{i} \rfloor$ cada. O conjunto destes conjuntos vai ser chamado $\bar{\mathcal{S}}$.

Se o algoritmo $\textsc{MinCC-Chvátal}$ sempre der prioridade para os elementos de $\bar{\mathcal{S}}$ quando os custos deles empatarem com os de $\mathcal{S}^*$, vai selecionar todos os elementos de $\bar{\mathcal{S}}$ e nenhum do outro conjunto, enquanto a solução ótima era exatamente oposta (selecionar todo $\mathcal{S}^*$ e nada mais). Portanto, a razão da aproximação encontrada pelo algoritmo é $|\bar{\mathcal{S}}|/|\mathcal{S}^*|$. Sabemos que $|\mathcal{S}^*| = n$, basta calcular $|\bar{\mathcal{S}}|$.

Pela descrição de $|\bar{\mathcal{S}}|$ sabemos quantos elementos existem em cada linha, então, podemos escrever
$$ |\bar{\mathcal{S}}| = \sum \limits_{i=1}^{n} \lfloor \frac{n}{i} \rfloor $$.
\end{proof}


\end{document}
