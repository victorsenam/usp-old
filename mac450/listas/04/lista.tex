%%%%%%%%%%%%%%%%%%%%%%%%%%%%%%%%%%%%%%%%%
% Programming/Coding Assignment
% LaTeX Template
%
% This template has been downloaded from:
% http://www.latextemplates.com
%
% Original author:
% Ted Pavlic (http://www.tedpavlic.com)
%
% Note:
% The \lipsum[#] commands throughout this template generate dummy text
% to fill the template out. These commands should all be removed when 
% writing assignment content.
%
% This template uses a Perl script as an example snippet of code, most other
% languages are also usable. Configure them in the "CODE INCLUSION 
% CONFIGURATION" section.
%
%%%%%%%%%%%%%%%%%%%%%%%%%%%%%%%%%%%%%%%%%

%----------------------------------------------------------------------------------------
%	PACKAGES AND OTHER DOCUMENT CONFIGURATIONS
%----------------------------------------------------------------------------------------

\documentclass{article}
\usepackage[utf8]{inputenc}
\usepackage[brazil]{babel}

\usepackage{amsmath,amsthm,amssymb,amsfonts} % Math stuff
%\usepackage{enumitem,tikz}
\usepackage{enumitem} % Better enumerates
\usepackage{dsfont}

\usepackage{fancyhdr} % Required for custom headers
\usepackage{lastpage} % Required to determine the last page for the footer
\usepackage{extramarks} % Required for headers and footers
\usepackage[usenames,dvipsnames]{color} % Required for custom colors
\usepackage{graphicx} % Required to insert images
\usepackage{listings} % Required for insertion of code
\usepackage{courier} % Required for the courier font
\usepackage{lipsum} % Used for inserting dummy 'Lorem ipsum' text into the template

% Margins
\topmargin=-0.45in
\evensidemargin=0in
\oddsidemargin=0in
\textwidth=6.5in
\textheight=9.0in
\headsep=0.25in

\linespread{1.1} % Line spacing

% Set up the header and footer
\pagestyle{fancy}
\lhead{\hmwkAuthorName} % Top left header
\chead{\hmwkClass: \hmwkTitle} % Top center head
\rhead{\firstxmark} % Top right header
\lfoot{\lastxmark} % Bottom left footer
\cfoot{} % Bottom center footer
\rfoot{Página\ \thepage\ de\ \protect\pageref{LastPage}} % Bottom right footer
\renewcommand\headrulewidth{0.4pt} % Size of the header rule
\renewcommand\footrulewidth{0.4pt} % Size of the footer rule

\setlength\parindent{0pt} % Removes all indentation from paragraphs

%----------------------------------------------------------------------------------------
%	DOCUMENT STRUCTURE COMMANDS
%	Skip this unless you know what you're doing
%----------------------------------------------------------------------------------------

% Header and footer for when a page split occurs within a problem environment
\newcommand{\enterProblemHeader}[1]{
\nobreak\extramarks{#1}{#1 continued on next page\ldots}\nobreak
\nobreak\extramarks{#1 (continued)}{#1 continued on next page\ldots}\nobreak
}

% Header and footer for when a page split occurs between problem environments
\newcommand{\exitProblemHeader}[1]{
\nobreak\extramarks{#1 (continued)}{#1 continued on next page\ldots}\nobreak
\nobreak\extramarks{#1}{}\nobreak
}

\setcounter{secnumdepth}{0} % Removes default section numbers
\newcounter{homeworkProblemCounter} % Creates a counter to keep track of the number of problems
\setcounter{homeworkProblemCounter}{10}

\newcommand{\homeworkProblemName}{}
\newenvironment{homeworkProblem}[1][\unskip]{ % Sets homework environment with adittional argument for extra naming
    \stepcounter{homeworkProblemCounter} % Increase counter for number of problems
    \renewcommand{\homeworkProblemName}{Problema \arabic{homeworkProblemCounter} #1} % Assign \homeworkProblemName the name of the problem
    \section{\homeworkProblemName} % Make a section in the document with the custom problem count
    \enterProblemHeader{\homeworkProblemName} % Header and footer within the environment
}{
\exitProblemHeader{\homeworkProblemName} % Header and footer after the environment
}

\newcommand{\problemAnswer}[1]{ % Defines the problem answer command with the content as the only argument
\noindent\framebox[\columnwidth][c]{\begin{minipage}{0.98\columnwidth}#1\end{minipage}} % Makes the box around the problem answer and puts the content inside
}

\newcommand{\homeworkSectionName}{}
\newenvironment{homeworkSection}[1]{ % New environment for sections within homework problems, takes 1 argument - the name of the section
\renewcommand{\homeworkSectionName}{#1} % Assign \homeworkSectionName to the name of the section from the environment argument
\subsection{\homeworkSectionName} % Make a subsection with the custom name of the subsection
\enterProblemHeader{\homeworkProblemName\ [\homeworkSectionName]} % Header and footer within the environment
}{
\enterProblemHeader{\homeworkProblemName} % Header and footer after the environment
}

%----------------------------------------------------------------------------------------
%	NAME AND CLASS SECTION
%----------------------------------------------------------------------------------------

\newcommand{\hmwkTitle}{Lista 4} % Assignment title
\newcommand{\hmwkDueDate}{21 de Setembro de 2016} % Due date
\newcommand{\hmwkClass}{MAC0450} % Course/class
\newcommand{\hmwkAuthorName}{Victor Sena Molero - 8941317} % Assignment author

\newcommand{\R}{\mathbb{R}}
\renewcommand{\S}{\mathbb{S}}

%----------------------------------------------------------------------------------------
%	TITLE PAGE
%----------------------------------------------------------------------------------------

\title{
\vspace{2in}
\textmd{\textbf{\hmwkClass:\ \hmwkTitle}}\\
\normalsize\vspace{0.1in}\small{\hmwkDueDate}\\
\vspace{3in}
}

\author{\textbf{\hmwkAuthorName}}
\date{} % Insert date here if you want it to appear below your name

%----------------------------------------------------------------------------------------

\begin{document}

%\maketitle

%----------------------------------------------------------------------------------------
%	TABLE OF CONTENTS
%----------------------------------------------------------------------------------------

%\setcounter{tocdepth}{1} % Uncomment this line if you don't want subsections listed in the ToC

%\newpage
%\tableofcontents
\newpage

\begin{homeworkProblem}
Aplique o método dual ao problema da cobertura mínima por conjuntos \textsc{(MinCC)}, definido na seção 2.2. Mostre que o algoritmo resultante é uma $\beta$-aproximação, onde $\beta$ é o número máximo de conjuntos em que um elemento aparece.

\begin{proof}[Resposta]
Primeiro, precisamos formular o primal e o dual do problema \textsc{(MinCC)}. Seja $E$ o conjunto de elementos do problema e $\mathcal{S}$ o conjunto de conjuntos. Seja, também $c \in \mathbb{Q}_{\geq}^{|\mathcal{S}|}$ o vetor de custos dos conjuntos em $\mathcal{S}$. Precisamos formular o primal e o dual usados na técnica dual nesta análise.  
\begin{align} \label{primal}
\min \quad        &c^Tx & \nonumber \\
\text{s.a.} \quad &x(\delta(e)) \geq 1 & \forall e \in E \\
                  &x_s \geq 0 & \forall s \in \mathcal{S} \nonumber
\end{align}

\begin{align} \label{dual}
\max \quad        &\mathds{1}^Ty & \nonumber \\
\text{s.a.} \quad &y(s) \leq c_s & \forall s \in \mathcal{S} \\
                  &y_e \geq 0 & \forall e \in E \nonumber
\end{align}
Com $y(s) = \sum \limits_{e \in s} y_e$ para todo $s \in \mathcal{S}$ e $x(\delta(e)) = \sum \limits_{s \in \delta(e)} x_s$ para todo $e \in E$.  
O vetor $\tilde{x}$ tal que $\tilde{x}_s = 1 \forall s \in \mathcal{S}$ é uma solução viável do primal enquanto o vetor nulo é viável no dual. Assim, vale o teorema da dualidade forte e soluções ótimas deste programa devem respeitar folgas complementares e, também, se $\bar{x}$ é ótimo no primal e $\bar{y}$ é ótimo no dual,

$$\mathrm{opt}(E, \mathcal{S}, c) \geq c^Tx = \mathds{1}^T \text{.}$$

Escolhemos, então, o conjunto $C$ de todos os conjuntos de $\mathcal{S}$ que respeitam $\bar{y}(s) = c_s$. Já que valem folgas complementares, se $\bar{x}_s > 0$, então o conjunto $s$ foi escolhido, logo, para todo $e \in E$, pelo menos um $s$ foi escolhido tal que $e \in s$, assim, $C$ é uma cobertura por conjuntos do conjunto $E$.  

Agora, temos que  
$$c(C) = \sum \limits_{s \in \mathcal{S}} c_s = \sum \limits_{s \in \mathcal{S}} \bar{y}(s) \text{, }$$
agora, se cada elemento aparece no máximo $\beta$ vezes em cada conjunto, temos que
$$ \sum \limits_{s \in \mathcal{S}} \bar{y}(s) \leq \beta \bar{y}{E} = \beta \mathrm{opt}(E, \mathcal, c) \text{.}$$

Assim, a estratégia apresentada é uma $\beta$-aproximação para o problema \textsc{MinCC}.
\end{proof}

\end{homeworkProblem}

\begin{homeworkProblem}
O \textsc{MinCA} é um caso particular "fácil" do \textsc{MinCC}: existe um algoritmo polinomial que o resolve. Mostre que o método dual dá uma $\Delta$-aproximação polinomial para o \textsc{MinCA}, onde $\Delta$ é o grau máximo em $G$.

\begin{proof}[Resposta]
É possível modelar uma instância de $\textsc{MinCA}(G, c)$ com uma instância de $\textsc{MinCC}(E, \mathcal{S}, c)$. Basta escolher $E = V(G)$, $\mathcal{S} = E(G)$ e usar o mesmo vetor de custos. Assim, temos que cada elemento de $E$ pertence a no máximo $\Delta$ conjuntos de $\mathcal{S}$. Assim, o método dual dá uma $\Delta$-aproximação para o problema. 
\end{proof}
\end{homeworkProblem}

\end{document}
