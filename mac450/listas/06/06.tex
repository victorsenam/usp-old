%%%%%%%%%%%%%%%%%%%%%%%%%%%%%%%%%%%%%%%%%
% Programming/Coding Assignment
% LaTeX Template
%
% This template has been downloaded from:
% http://www.latextemplates.com
%
% Original author:
% Ted Pavlic (http://www.tedpavlic.com)
%
% Note:
% The \lipsum[#] commands throughout this template generate dummy text
% to fill the template out. These commands should all be removed when 
% writing assignment content.
%
% This template uses a Perl script as an example snippet of code, most other
% languages are also usable. Configure them in the "CODE INCLUSION 
% CONFIGURATION" section.
%
%%%%%%%%%%%%%%%%%%%%%%%%%%%%%%%%%%%%%%%%%

%----------------------------------------------------------------------------------------
%	PACKAGES AND OTHER DOCUMENT CONFIGURATIONS
%----------------------------------------------------------------------------------------

\documentclass{article}
\usepackage[utf8]{inputenc}
\usepackage[brazil]{babel}

\usepackage{amsmath,amsthm,amssymb,amsfonts} % Math stuff
%\usepackage{enumitem,tikz}
\usepackage{enumitem} % Better enumerates
\usepackage{dsfont}

\usepackage{fancyhdr} % Required for custom headers
\usepackage{lastpage} % Required to determine the last page for the footer
\usepackage{extramarks} % Required for headers and footers
\usepackage[usenames,dvipsnames]{color} % Required for custom colors
\usepackage{graphicx} % Required to insert images
\usepackage{listings} % Required for insertion of code
\usepackage{courier} % Required for the courier font
\usepackage{lipsum} % Used for inserting dummy 'Lorem ipsum' text into the template

% Margins
\topmargin=-0.45in
\evensidemargin=0in
\oddsidemargin=0in
\textwidth=6.5in
\textheight=9.0in
\headsep=0.25in

\linespread{1.1} % Line spacing

% Set up the header and footer
\pagestyle{fancy}
\lhead{\hmwkAuthorName} % Top left header
\chead{\hmwkClass: \hmwkTitle} % Top center head
\rhead{\firstxmark} % Top right header
\lfoot{\lastxmark} % Bottom left footer
\cfoot{} % Bottom center footer
\rfoot{Página\ \thepage\ de\ \protect\pageref{LastPage}} % Bottom right footer
\renewcommand\headrulewidth{0.4pt} % Size of the header rule
\renewcommand\footrulewidth{0.4pt} % Size of the footer rule

\setlength\parindent{0pt} % Removes all indentation from paragraphs

%----------------------------------------------------------------------------------------
%	DOCUMENT STRUCTURE COMMANDS
%	Skip this unless you know what you're doing
%----------------------------------------------------------------------------------------

% Header and footer for when a page split occurs within a problem environment
\newcommand{\enterProblemHeader}[1]{
\nobreak\extramarks{#1}{#1 continua na próxima página\ldots}\nobreak
\nobreak\extramarks{#1 (continua)}{#1 continua na próxima página\ldots}\nobreak
}

% Header and footer for when a page split occurs between problem environments
\newcommand{\exitProblemHeader}[1]{
\nobreak\extramarks{#1 (continua)}{#1 continua na próxima página\ldots}\nobreak
\nobreak\extramarks{#1}{}\nobreak
}

\setcounter{secnumdepth}{0} % Removes default section numbers
\newcounter{homeworkProblemCounter} % Creates a counter to keep track of the number of problems

\newcommand{\homeworkProblemName}{}
\newenvironment{homeworkProblem}[1][\unskip]{ % Sets homework environment with adittional argument for extra naming
    \stepcounter{homeworkProblemCounter} % Increase counter for number of problems
    \renewcommand{\homeworkProblemName}{Problema \arabic{homeworkProblemCounter} #1} % Assign \homeworkProblemName the name of the problem
    \section{\homeworkProblemName} % Make a section in the document with the custom problem count
    \enterProblemHeader{\homeworkProblemName} % Header and footer within the environment
}{
\exitProblemHeader{\homeworkProblemName} % Header and footer after the environment
}

\newcommand{\homeworkSectionName}{}
\newenvironment{homeworkSection}[1]{ % New environment for sections within homework problems, takes 1 argument - the name of the section
\renewcommand{\homeworkSectionName}{#1} % Assign \homeworkSectionName to the name of the section from the environment argument
\subsection{\homeworkSectionName} % Make a subsection with the custom name of the subsection
\enterProblemHeader{\homeworkProblemName\ [\homeworkSectionName]} % Header and footer within the environment
}{
\enterProblemHeader{\homeworkProblemName} % Header and footer after the environment
}

\newenvironment{homeworkProblemAnswer}[0]{ % Defines the problem answer environment
    \subsection*{Resposta}
    }{
}

%----------------------------------------------------------------------------------------
%	THEOREMS AND COUNTERS
%----------------------------------------------------------------------------------------

\numberwithin{equation}{homeworkProblemCounter}

\newtheorem{prop}[equation]{Proposição}

%----------------------------------------------------------------------------------------
%	NAME AND CLASS SECTION
%----------------------------------------------------------------------------------------

\newcommand{\hmwkTitle}{Prova 2} % Assignment title
\newcommand{\hmwkDueDate}{22 de Outubro de 2016} % Due date
\newcommand{\hmwkClass}{MAC0343} % Course/class
\newcommand{\hmwkAuthorName}{Victor Sena Molero - 8941317} % Assignment author

%----------------------------------------------------------------------------------------
%	TITLE PAGE
%----------------------------------------------------------------------------------------

\title{
\vspace{2in}
\textmd{\textbf{\hmwkClass:\ \hmwkTitle}}\\
\normalsize\vspace{0.1in}\small{\hmwkDueDate}\\
\vspace{3in}
}

\author{\textbf{\hmwkAuthorName}}
\date{} % Insert date here if you want it to appear below your name

%----------------------------------------------------------------------------------------
%	TABLE OF CONTENTS
%----------------------------------------------------------------------------------------

%\setcounter{tocdepth}{1} % Uncomment this line if you don't want subsections listed in the ToC


\setcounter{homeworkProblemCounter}{2}

%----------------------------------------------------------------------------------------
%>-- SHORTCUTS
%----------------------------------------------------------------------------------------

\renewcommand{\|}[1]{\mathbb{#1}}
\renewcommand{\"}[1]{\ensuremath{\mathcal{#1}}}

\newcommand{\sse}{\Leftrightarrow}
\newcommand{\se}{\Rightarrow}
\newcommand{\so}{\Leftarrow}

\newcommand{\eigen}{\lambda^{\downarrow}}
\newcommand{\Tr}{\textrm{Tr}}

%----------------------------------------------------------------------------------------

\begin{document}

%\maketitle
%\newpage

%\tableofcontents
%\newpage

\begin{homeworkProblem}[]
Apresente uma versão desaleatorizada da $0.5$-aproximação probabilística apresentada para o problema do $\textsc{MaxCut}(V, E, w)$.

\begin{homeworkProblemAnswer}
Primeiro, vamos definir o custo de um corte $\emptyset \neq S \subset V$ por
$$ c(S) = \sum \limits_{\substack{uv \in E\\ u \in S \\ v \notin S}} w_{uv}. $$

Agora, definimos a variável aleatória $X_S$ como o valor do corte de um conjunto $S$ escolhido pelo algoritmo criado no item anterior. Temos que a esperança de $X_S$ é dada por:
$$ E[X_S] = \sum \limits_{\emptyset \neq S \subset V} c(S). $$

Nosso algoritmo vai gerar um conjunto $S$ que define o corte final. Para isso, vamos manter dois conjuntos $\hat{W}$ e $\hat{S}$, que representam, respectivamente, os elementos já processados pelo algoritmo e os elementos para os quais se escolheu pertencer a $S$. Isto é, ao último passo do algoritmo, teremos $\hat{W} = V$ e $S$ será definido pelo $\hat{S}$ e retornado pelo algoritmo.  

Vamos considerar um passo qualquer do algoritmo, ou seja, temos dois conjuntos $\hat{S}$ e $\hat{W}$ como definidos acima e queremos escolher um $v \in V \setminus \hat{W}$. Temos
$$ E[X_S \mid \hat{W} \cap S = \hat{S}] = 0.5 * E[X_S \mid (\hat{W} \cup v) \cap S = \hat{S}] + 0.5 * E[X_S \mid (\hat{W} \cup v) \cap S = \hat{S} \cup v].$$

Portanto, 
\begin{equation} \label{eq:desigualdade_esperancas}
    \max(E[X_S \mid (\hat{W} \cup v) \cap S = \hat{S}], E[X_S \mid (\hat{W} \cup v) \cap S = \hat{S} \cup v]) \geq E[X_S \mid \hat{W} \cap S = \hat{S}].
\end{equation}

Se pensarmos em esperanças condicionadas, podemos simplesmente escolher se $v \in S$ ou $v \notin S$ de acordo com a escolha que maximize $E[X_S | \hat{W} \cap S = \hat{S}]$. Pela desigualdade \eqref{eq:desigualdade_esperancas}, a cada passo, esta esperança só aumenta. Portanto, se $\bar{W}$ e $\bar{S}$ são os $\hat{W}$ e $\hat{S}$ criados após o último passo do algorimto, temos

$$ E[X_S \mid (\bar{W} \cap S) = \bar{S}] = E[X_S \mid (V \cap S) = \hat{S}] = E[X_S \mid S = \hat{S}] = c(\hat{S}),$$
porém, já que sempre escolhemos a opção que maximiza a esperança para formar $\hat{S}$ e $\hat{W}$, temos, por \eqref{eq:desigualdade_esperancas},
$$ E[X_S \mid (\bar{W} \cap S) = \bar{S}] \geq E[X_S \mid (\hat{W} \cap S) = \hat{S}] $$
para quaisquer $\hat{W}$ e $\hat{S}$ gerados em qualquer passo do algoritmo. Já que, antes do primeiro passo, $\hat{W} = \hat{S} = \emptyset$, temos, para qualquer solução ótima $S^*$, 
$$ 
    E[X_S \mid (\bar{S} \cap S) = 
    \bar{S}] \geq E[X_S \mid (\emptyset \cap S) = \emptyset] = 
    E[X_S] = 
    \sum \limits_{uv \in E} 0.5w_{uv} =
    \sum \limits_{\substack{uv \in E \\ u \in S^* \\ v \notin S^*}} 0.5w_{uv} = 
    0.5opt(V,E,w).
$$
Portanto, sempre escolher a opção que maximize a esperança, ou seja, seguir o método das esperanças condicionadas, gera uma $0.5$-aproximação para o problema $\textsc{MaxCut}(V,E,w)$. Podemos escrever o pseudo-código deste algoritmo agora. Primeiro, vamos escrever uma função auxiliar $\textsc{CalcEsp}(\hat{W},\hat{S}, V, E, w)$ que calcula o valor $E[X_S \mid \hat{W} \cap S = \hat{S}]$.

\pagebreak

\begin{algorithmic}[1]
\Function{CalcEsp}{\hat{W}, \hat{S}, V, E, w}
\State $r \leftarrow 0$
    \For{$uv \in E$}
        \If{$u \in \bar{W}$ e $v \in \bar{W}$}
            \If{$u \in \bar{S}$ xor $v \in \bar{S}$}
                \State $r \leftarrow r + w_{uv}$
            \EndIf
        \ElsIf{$u \in \bar{W}$ ou $v \in \bar{W}$}
            \State $r \leftarrow r + \frac{1}{2}w_{uv}$
        \EndIf
    \EndFor
\State \Return r
\EndFunction
\end{algorithmic}

E, agora, podemos escrever o algoritmo $\textsc{MaxCut-Dealeatorizado}(V, E, w)$ que utiliza $\textsc{CalcEsp}$ como função auxiliar.

\begin{algorithmic}[1]
\Function{MaxCut-Dealeatorizado}{V, E, w}
\State $\hat{W} \leftarrow \emptyset$
\State $\hat{S} \leftarrow \emptyset$
    \For{$v \in V$}
        \State $\hat{W} \leftarrow \hat{W} \cup v$
        \State $a \leftarrow \textsc{CalcEsp}(\hat{W}, \hat{S} \cup v, V, E, w)$
        \State $b \leftarrow \textsc{CalcEsp}(\hat{W}, \hat{S}, V, E, w)$
        \If{$a \geq b$}
            \State $\hat{S} \leftarrow \hat{S} \cup v$
        \EndIf
    \EndFor
\State \Return $\hat{S}$
\EndFunction
\end{algorithmic}

\end{homeworkProblemAnswer}

\end{homeworkProblem}

\end{document}
