\documentclass[a4paper]{article}

\usepackage[english]{babel}
\usepackage[utf8]{inputenc}
\usepackage{amsmath}
\usepackage{graphicx}
\usepackage[colorinlistoftodos]{todonotes}

\title{Resumo Cap. 1}

\author{Victor Sena Molero - 8941317}

\date{\today}

\begin{document}
\maketitle

O primeiro capítulo do livro traça contraste entre aplicações SaaS de sucesso e grandes fracassos. Enfatizando a potencial importancia de uma aplicação do tipo e o impacto que elas tem na sociedade quando fracassam ou quando têm sucesso com o objetivo de motivar o estudo do livro e dos motivos que levam ao sucesso ou fracasso de uma aplicação do tipo. \ 
Agora o livro mostra e também contrasta duas formas diferentes de se planejar e executar um projeto de engenharia de software, dois "ciclos de vida": O ciclo em cascata e o ciclo em espiral. Ao final, é apresentado uma nova organização para a engenharia de Software, o "Processo Unificado da Rational", que combina os benefícios dos dois ciclos baseando-se num plano de 4 etapas e 6 "disciplinas de engenharia". \ 
É apresentado o conceito de desenvolvimento ágil que se diferencia das técnicas de desenvolvimento de Software antes apresentadas por ser mais integrada ao cliente. O método ágil se beneficia de pensar no projeto como uma ideia que cresce junto com o projeto. Que implementa ideias e histórias novas durante o desenvolvimento do projeto. Onde o cliente pode ter liberdade para alterar o produto enquanto ele é desenvolvido progressivamente e vai criando novas funcionalidades. \ 
A arquitetura orientada a serviço (SOA) é introduzida como uma forma de modularizar o software e fazer com que as funcionalidades de um software sejam vistas como subserviços independentes e integrados. Isso se encaixa muito bem na ideia de desenvolvimento ágil, pois faz com que seja possível criar novos serviços não planejados sobre antigos de forma elegante. \ 

\end{document}
