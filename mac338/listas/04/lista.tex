\documentclass[12pt]{article}
\usepackage[utf8]{inputenc}
\usepackage[margin=1in]{geometry} 
\usepackage{amsmath,amsthm,amssymb,amsfonts,algpseudocode}
 
\newcommand{\N}{\mathbb{N}}
\newcommand{\Z}{\mathbb{Z}}
 
\newenvironment{problem}[2][Ex]{\begin{trivlist}
\item[\hskip \labelsep {\bfseries #1}\hskip \labelsep {\bfseries #2.}]}{\end{trivlist}}
%If you want to title your bold things something different just make another thing exactly like this but replace "problem" with the name of the thing you want, like theorem or lemma or whatever
 
\begin{document}
 
%\renewcommand{\qedsymbol}{\filledbox}
%Good resources for looking up how to do stuff:
%Binary operators: http://www.access2science.com/latex/Binary.html
%General help: http://en.wikibooks.org/wiki/LaTeX/Mathematics
%Or just google stuff
 
\title{Lista 4}
\author{Victor Sena Molero - 8941317}
\maketitle
 
\begin{problem}{1}
Escreva uma função que recebe um vetor com $n$ letras A's e B's e, por meio de trocas, move todos os A's para o início do vetor. Sua função deve consumir tempo $O(n)$. 
\end{problem}
 
\begin{proof}[Algoritmo]
\begin{algorithmic}
\\
\State $v \gets$ entrada
\State $c \gets 1$
\State $i \gets 1$ 
\While {$i \leq n$}
    \If{$v[i] = \text{A}$}
        \State $v[i] \gets v[c]$
        \State $v[c] \gets \text{A}$
        \State $c \gets c+1$
    \EndIf
    \State $i \gets i+1$
\EndWhile
\end{algorithmic}
\end{proof}

\begin{problem}{3}
Sejam $X[1\dots n]$ e $Y[1\dots n]$ dois vetores, cada um contendo $n$ números ordenados. Escreva um algoritmo $O(\lg n)$ para encontrar uma das medianas de todos os $2n$ elementos nos vetores $X$ e $Y$. 
\end{problem}

\begin{proof}[Algoritmo]
\begin{algorithmic}
\\
\State $X,Y \gets$ entrada
\State $lo \gets 1$
\State $hi \gets n$
\While {$lo < hi$}
    \State $mid \gets lo+(hi-lo)/2$
    \If {$X[mid] < Y[n-mid+1]$}
        \State $lo \gets mid+1$
    \Else
        \State $hi \gets mid$
    \EndIf
\EndWhile
\State \Return $X[lo]$

\end{algorithmic}
Eu sei que falta algo nesse algoritmo, não consegui arrumar, ainda.
\end{proof}

\end{document}
