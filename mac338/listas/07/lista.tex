\documentclass[12pt]{article}
\usepackage[utf8]{inputenc}
\usepackage[margin=1in]{geometry} 
\usepackage[brazil]{babel}
\usepackage{amsmath,amsthm,amssymb,amsfonts,listings,algpseudocode}
 
\newcommand{\N}{\mathbb{N}}
\newcommand{\Z}{\mathbb{Z}}
 
\newcounter{exCounter}
\setcounter{exCounter}{22}
\newtheorem{lema}{Lema}
\newtheorem{ex}[exCounter]{Ex}
\newenvironment{problem}[2][Ex]{\begin{trivlist}
\item[\hskip \labelsep {\bfseries #1}\hskip \labelsep {\bfseries #2.}]}{\end{trivlist}}
%If you want to title your bold things something different just make another thing exactly like this but replace "problem" with the name of the thing you want, like theorem or lemma or whatever
 
\begin{document}
 
%\renewcommand{\qedsymbol}{\filledbox}
%Good resources for looking up how to do stuff:
%Binary operators: http://www.access2science.com/latex/Binary.html
%General help: http://en.wikibooks.org/wiki/LaTeX/Mathematics
%Or just google stuff
 
\title{Lista 7}
\author{Victor Sena Molero - 8941317}
\maketitle

\begin{ex}
Festival de Estátuas de Gelo
\end{ex}

\begin{proof}[Resposta]
Dado um $m$ e um vetor $a$ de $n$ inteiros $a_1, a_2, \dots, a_n$, temos que encontrar um vetor $x \in \mathbb{N}^n$ que minimize $\sum\limits_{i=1}^n x_i$ sujeito a $x^Ta = m$. Ou seja, devemos conseguir somar os inteiros dados no vetor $a$ de forma a atingir um valor exatamente $m$, podendo usar cada valor de $a$ quantas vezes forem necessárias e minimzando a quantidade de somas feitas. Vamos chamar a solução ótima de uma instância com um $m$ dado de $f(m)$.

Temos que $f(0) = 0$ pois não precisamos somar nenhum inteiro para atingir o valor 0. Agora, queremos descobrir o valor de $f(m)$ dados os valores de todas as instâncias menores, ou seja, sabendo os valores para todo $f(x)$ tal que $x < m$. Sabemos que, se temos um vetor de $a$ e queremos atingir a soma $m$, devemos escolher algum valor de $a$ e colocar nesta soma, porém, após inserir este valor $x$, temos que conseguir somar $m-x$ com os mesmos valores $a$ do problema anterior, ou seja, este é um subproblema do problema anterior. 

Podemos, então, concluir a seguinte recorrência sobre um valor de $f(m)$ quando $m > 0$
$$ f(m) = \min\limits_{i=1}^{n} f(m-a_i)+1 $$

Para isso, podemos criar uma tabela que memoriza o resultado da função para todo $m$ e calcular os valores dela crescentemente em $m$, sabendo que $f(0) = 0$, ou seja, podemos aplicar programação dinâmica.

Para calcular cada um dos valores, precisamos iterar por todo o vetor $a$, ou seja, cada cálculo de estado custa $O(n)$. Já que temos $m$ estados para calcular, o tempo total de execução do programa será $O(nm)$.

Eu tive minha submissão aceita no URI, meu user lá é Victor Sena Molero. Segue o código submetido no juíz, em C.
\lstinputlisting[language=C++]{gelo.cpp}
\end{proof}

\begin{ex}
LISA - Pocket Money
\end{ex}

\begin{proof}[Resposta]

\end{document}
