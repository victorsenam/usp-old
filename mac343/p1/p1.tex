%%%%%%%%%%%%%%%%%%%%%%%%%%%%%%%%%%%%%%%%%
% Programming/Coding Assignment
% LaTeX Template
%
% This template has been downloaded from:
% http://www.latextemplates.com
%
% Original author:
% Ted Pavlic (http://www.tedpavlic.com)
%
% Note:
% The \lipsum[#] commands throughout this template generate dummy text
% to fill the template out. These commands should all be removed when 
% writing assignment content.
%
% This template uses a Perl script as an example snippet of code, most other
% languages are also usable. Configure them in the "CODE INCLUSION 
% CONFIGURATION" section.
%
%%%%%%%%%%%%%%%%%%%%%%%%%%%%%%%%%%%%%%%%%

%----------------------------------------------------------------------------------------
%	PACKAGES AND OTHER DOCUMENT CONFIGURATIONS
%----------------------------------------------------------------------------------------

\documentclass{article}
\usepackage[utf8]{inputenc}
\usepackage[brazil]{babel}

\usepackage{amsmath,amsthm,amssymb,amsfonts} % Math stuff
%\usepackage{enumitem,tikz}
\usepackage{enumitem} % Better enumerates
\usepackage{dsfont}

\usepackage{fancyhdr} % Required for custom headers
\usepackage{lastpage} % Required to determine the last page for the footer
\usepackage{extramarks} % Required for headers and footers
\usepackage[usenames,dvipsnames]{color} % Required for custom colors
\usepackage{graphicx} % Required to insert images
\usepackage{listings} % Required for insertion of code
\usepackage{courier} % Required for the courier font
\usepackage{lipsum} % Used for inserting dummy 'Lorem ipsum' text into the template

% Margins
\topmargin=-0.45in
\evensidemargin=0in
\oddsidemargin=0in
\textwidth=6.5in
\textheight=9.0in
\headsep=0.25in

\linespread{1.1} % Line spacing

% Set up the header and footer
\pagestyle{fancy}
\lhead{\hmwkAuthorName} % Top left header
\chead{\hmwkClass: \hmwkTitle} % Top center head
\rhead{\firstxmark} % Top right header
\lfoot{\lastxmark} % Bottom left footer
\cfoot{} % Bottom center footer
\rfoot{Page\ \thepage\ of\ \protect\pageref{LastPage}} % Bottom right footer
\renewcommand\headrulewidth{0.4pt} % Size of the header rule
\renewcommand\footrulewidth{0.4pt} % Size of the footer rule

\setlength\parindent{0pt} % Removes all indentation from paragraphs

%----------------------------------------------------------------------------------------
%	DOCUMENT STRUCTURE COMMANDS
%	Skip this unless you know what you're doing
%----------------------------------------------------------------------------------------

% Header and footer for when a page split occurs within a problem environment
\newcommand{\enterProblemHeader}[1]{
\nobreak\extramarks{#1}{#1 continued on next page\ldots}\nobreak
\nobreak\extramarks{#1 (continued)}{#1 continued on next page\ldots}\nobreak
}

% Header and footer for when a page split occurs between problem environments
\newcommand{\exitProblemHeader}[1]{
\nobreak\extramarks{#1 (continued)}{#1 continued on next page\ldots}\nobreak
\nobreak\extramarks{#1}{}\nobreak
}

\setcounter{secnumdepth}{0} % Removes default section numbers
\newcounter{homeworkProblemCounter} % Creates a counter to keep track of the number of problems

\newcommand{\homeworkProblemName}{}
\newenvironment{homeworkProblem}[1][\unskip]{ % Sets homework environment with adittional argument for extra naming
    \stepcounter{homeworkProblemCounter} % Increase counter for number of problems
    \renewcommand{\homeworkProblemName}{Problema \arabic{homeworkProblemCounter} #1} % Assign \homeworkProblemName the name of the problem
    \section{\homeworkProblemName} % Make a section in the document with the custom problem count
    \enterProblemHeader{\homeworkProblemName} % Header and footer within the environment
}{
\exitProblemHeader{\homeworkProblemName} % Header and footer after the environment
}

\newcommand{\problemAnswer}[1]{ % Defines the problem answer command with the content as the only argument
\noindent\framebox[\columnwidth][c]{\begin{minipage}{0.98\columnwidth}#1\end{minipage}} % Makes the box around the problem answer and puts the content inside
}

\newcommand{\homeworkSectionName}{}
\newenvironment{homeworkSection}[1]{ % New environment for sections within homework problems, takes 1 argument - the name of the section
\renewcommand{\homeworkSectionName}{#1} % Assign \homeworkSectionName to the name of the section from the environment argument
\subsection{\homeworkSectionName} % Make a subsection with the custom name of the subsection
\enterProblemHeader{\homeworkProblemName\ [\homeworkSectionName]} % Header and footer within the environment
}{
\enterProblemHeader{\homeworkProblemName} % Header and footer after the environment
}

%----------------------------------------------------------------------------------------
%	NAME AND CLASS SECTION
%----------------------------------------------------------------------------------------

\newcommand{\hmwkTitle}{Prova 1} % Assignment title
\newcommand{\hmwkDueDate}{20 de Setembro de 2016} % Due date
\newcommand{\hmwkClass}{MAC0343} % Course/class
\newcommand{\hmwkAuthorName}{Victor Sena Molero - 8941317} % Assignment author

\newcommand{\R}{\mathbb{R}}
\renewcommand{\S}{\mathbb{S}}

%----------------------------------------------------------------------------------------
%	TITLE PAGE
%----------------------------------------------------------------------------------------

\title{
\vspace{2in}
\textmd{\textbf{\hmwkClass:\ \hmwkTitle}}\\
\normalsize\vspace{0.1in}\small{\hmwkDueDate}\\
\vspace{3in}
}

\author{\textbf{\hmwkAuthorName}}
\date{} % Insert date here if you want it to appear below your name

%----------------------------------------------------------------------------------------

\begin{document}

\maketitle

%----------------------------------------------------------------------------------------
%	TABLE OF CONTENTS
%----------------------------------------------------------------------------------------

%\setcounter{tocdepth}{1} % Uncomment this line if you don't want subsections listed in the ToC

%\newpage
%\tableofcontents
\newpage

\begin{homeworkProblem}
Seja $A \in R^{m \times n}$ uma matriz. Prove que vale precisamente uma das seguintes alternativas:
\begin{enumerate}[label={\roman*}]
\item existe $x \in \mathbb{R}_+^n$ tal que $Ax = 0$ e $x \neq 0$;
\item existe $y \in \mathbb{R}^m$ tal que $A^Ty < 0$.
\end{enumerate}

\begin{proof}[Resposta]
Queremos provar que vale exatamente um entre $(i)$ e $(ii)$. Para provar isso, vamos considerar um PL e seu dual.  

\begin{align} \label{primal}
\min \quad        &c^Tx \nonumber \\
\text{s.a.} \quad &Ax=b \\
                  &x \in \mathbb{R}_+^n \nonumber
\end{align}

\begin{align} \label{dual}
\max \quad        &b^Ty \nonumber \\
\text{s.a.} \quad &y\in \mathbb{R}^m \\
                  &A^Ty \leq c \nonumber
\end{align}

Vamos provar, primeiro que vale pelo menos um entre $(i)$ e $(ii)$. Assuma, para a matriz $A$ de $\ref{primal}$ e $\ref{dual}$ que não vale $(i)$, então, no programa $\ref{primal}$, $x = 0$ é o único ponto viável. Temos, então, que para todo $c \in \mathbb{R}^n$, $\ref{primal}$ é viável e tem solução ótima $\min{c^Tx} = 0$.  

Pelo teorema 12 (Dualidade Forte de PL), segue que $\ref{dual}$ é viável, portanto, existe $A^Ty \leq c$. Basta escolher $c < 0$ e temos que $A^Ty < 0$, ou seja, vale $(ii)$.  

Agora, vamos provar que vale no máximo 1 entre $(i)$ e $(ii)$. Assuma que vale $(i)$ e $(ii)$, então, existe $y \in \mathbb{R}^m$ tal que $A^Ty < 0$, escolhemos, nos programas $\ref{primal}$ e $\ref{dual}$, $b = 0$ e $c = A^Ty$, assim, temos que $\ref{dual}$ é viável. Além disso, já que vale $(i)$, existe $0 \neq x \in \mathbb{R}_+^n$.  

Pelo teorema 7 (Dualidade Fraca de PL), $c^tx \geq b^Ty = 0^Ty = 0$, porém, já que $c < 0$ e $0 \neq x \geq 0$, $c^Tx < 0$, uma contradição. Com isso, concluímos que vale exatamente 1 dentre $(i)$ e $(ii)$.
\end{proof}

\end{homeworkProblem}

\begin{homeworkProblem}
Sejam $X,S \in \S^n$. Prove que $0 \prec S \preceq X \Rightarrow 0 \prec X^{-1} \preceq S^{-1}$.
\end{homeworkProblem}

\begin{homeworkProblem}
Sejam $X,S \in \S^n$. Prove que
\begin{enumerate}[label={\roman*}]
\item $X,S \in \S_+^n \Longrightarrow X \circ S \succeq 0$;

\begin{proof}[Resposta]
    Se $S \succeq 0$, pelo Teorema 20 (item $(iii)$), existe uma matriz $H \in R^{m \times n}$ para algum $m$ e um vetor $s \in \R_+^m$ tal que $S = \sum \limits_{i=1}^{m} s_i(He_i)(He_i)^T$. Escolhemos tais $H$ e $s$, assim, para todo $q \in \R^n$, temos que
    $$q^T(X \circ S)q = \sum \limits_{i=1}^n \sum \limits_{j=1}^n q_iq_jX_{i,j}S_{i,j} =
    \sum \limits_{i=1}^n \sum \limits_{j=1}^n (q_iq_jX_{i,j} \sum \limits_{k=1}^m s_kH_{k,i}H_{k,j}) \text{, que pode ser escrito como }$$
    $$\sum \limits_{k=1}^m s_k \sum \limits_{i=1}^n \sum \limits_{j=1}^n (q \circ (He_k))_{i} (q \circ (He_k)_{j} X_{i,j} \text{, }$$
    Se definirmos a matriz $Q = q \mathds{1}^T$, temos,
    $$q^T(X \circ S)q = \sum \limits_{k=1}^m s_k((Q \circ H)e_k)^T X (Q \circ H)e_k \text{, }$$
    já que para todo $k \in [m]$, vale que $(Q \circ H)e_k \in \R^n$ e $s_k \geq 0$, e, além disso, $X \succeq 0$,
    $$q^T(X \circ S)q \geq \sum \limits_{k=1}^m s_k 0 \geq 0 \text{.}$$
    Portanto, $(X \circ S) \succeq 0$.
\end{proof}

\item $X,S \in \S_{++}^n \Longrightarrow X \circ S \succ 0$;

\begin{proof}[Resposta]
    Se $S \succ 0$, pelo Exercício 21 (item $(iii)$), existe uma matriz $H \in R^{m \times n}$ para algum $m$ e um vetor $s \in \R_{++}^m$ tal que $S = \sum \limits_{i=1}^m s_i(He_i)(He_i)^T$ e o $\mathrm{span}(\{He_k \mid k \in [m]\}) = \R^n$. Escolhemos tais $H$ e $s$, assim, para todo $q \in \R^n$, temos que
    $$q^T(X \circ S)q = \sum \limits_{i=1}^n \sum \limits_{j=1}^n q_iq_jX_{i,j}S_{i,j} =
    \sum \limits_{i=1}^n \sum \limits_{j=1}^n (q_iq_jX_{i,j} \sum \limits_{k=1}^m s_kH_{k,i}H_{k,j}) \text{, que pode ser escrito como }$$
    $$\sum \limits_{k=1}^m s_k \sum \limits_{i=1}^n \sum \limits_{j=1}^n (q \circ (He_k))_{i} (q \circ (He_k)_{j} X_{i,j} \text{, }$$
    Se definirmos a matriz $Q = q \mathds{1}^T$, temos,
    $$q^T(X \circ S)q = \sum \limits_{k=1}^m s_k((Q \circ H)e_k)^T X (Q \circ H)e_k \text{, }$$
    já que para todo $k \in [m]$, vale que $(Q \circ H)e_k \in \R^n$ e $s_k > 0$, e, além disso, $X \succ 0$,
    $$q^T(X \circ S)q > \sum \limits_{k=1}^m s_k 0 > 0 \text{.}$$
    Portanto, $(X \circ S) \succ 0$.
\end{proof}

\item $X,S \in \S_{++}^n \Longrightarrow X \circ S \succeq (X^{-1} \circ S^{-1})^{-1}$;
\item $X,S \in \S_{++}^n \Longrightarrow X \circ S \succeq (X \circ S)^{-1}$.
\end{enumerate}
\end{homeworkProblem}

\begin{homeworkProblem}
Seja $n$ um inteiro positivo. Determine o valor ótimo do seguinte programa semidefinido:

\begin{align} \label{ex4}
\max \quad        &\mathds{1}^T + 4z_1 & \nonumber \\
\text{s.a.} \quad &y \in \mathbb{R}^n \text{, } & \nonumber \\
                  &z \in \mathbb{R}^{n+1} \text{, } & \\
                  &
                  \begin{bmatrix}
                    -y_j & -z_j \\
                    -z_j & 2z_{j+1}
                  \end{bmatrix}
                  \succeq a\text{, } & \forall j \in [n] \text{, } \nonumber \\
                  &z_{n+1} = \frac{1}{2} & \nonumber
\end{align}

%\begin{proof}[Resposta]
%Para todo $j \in [n]$, $A_j \succeq 0$ se e somente se para todo $h \in \R^2$, $h^TA_jh \geq 0$, então,
%$$ h^TA_jh = -y_jh_1^2 - 2z_jh_1h_2 + 2z_{j+1}h_2^2 \geq 0 \text{.}$$
%\end{proof}

\end{homeworkProblem}

\begin{homeworkProblem}
Seja $\gamma \in \mathbb{R}_{++}$. Considere o programa semidefinido

\begin{align} \label{ex5}
\max \quad        &\langle C, X \rangle & \nonumber \\
\text{s.a.} \quad &\langle A_i, X \rangle \text{, } & \forall i \in [m] \text{, } \\
                  &x \in \S_+^n \text{, } \nonumber
\end{align}
onde $n := m := 3$,  
$$  C := e_1e_2^T + e_2e_1^T \text{, } 
    A_1 := e_2e_2^T \text{, }
    A_2 := e_1e_3^T + e_3e_1^T \text{, }
    A_3 := -C + 2e_3e_3^T \text{ e }
    b := 2 \gamma e_3 $$

\end{homeworkProblem}

\begin{homeworkProblem}
Seja $\emptyset \neq K \subseteq \mathbb{E}$ um cone convexo e fechado num espaço euclidiano. Prove que $K^{**} = K$.
\end{homeworkProblem}

\begin{homeworkProblem}
Seja $X \in \S^n$. Prove que $X \succ 0 \Longleftrightarrow det(X[\{1,\dots,k\}]) > 0$ para todo $k \in [n]$.

\begin{proof}[Resposta]
Seja $X \in \S^n$, vamos provar a tese sugerida pelo enunciado por indução em $n$. Se $n=1$, temos que $X \succ 0 \Leftrightarrow X > 0 \Leftrightarrow det(X) > 0$. Agora tome por hipótese de indução que a tese vale para $n - 1$.  

Seja, então, $X \in \S^n$. $X$ pode ser decomposto da seguinte maneira:
$$X = \begin{bmatrix}
    \bar{X} y\\
    y^T \alpha
\end{bmatrix}
\text{.}$$  

Vamos provar a ida, ou seja, assuma que $X \succ 0$, para todo $h \in \R^n$, $h^TXh > 0$, podemos definir 
$$h = \begin{bmatrix}
    \bar{h}\\
    0
\end{bmatrix} \text{,}$$
temos que $h^TXh = \bar{h}^T\bar{X}\bar{h} > 0$ para todo $\bar{h} \in \R^{n-1}$, logo, $\bar{X} \succ 0$. Assim, $det(X[\{1,\dots,k\}]) > 0$ para todo $k \in [n-1]$. Além disso, $det(X) = det(\bar{X})det(\alpha - y^T\bar{X}y)$. Pelo Ex. 18 (Complemento de Schur), $\alpha - y^T\bar{X}y > 0$, já que $X \succ 0$. Assim, $det(X) > 0$. Logo, provamos que $X \succ 0 \Longrightarrow det(X[\{1,\dots,k\}]) > 0$ para todo $k \in [n]$.  

Agora precisamos assumir que $det(X[\{1,\dots,k\}]) > 0$ e provar que $X \succ 0$. Mais uma vez, usaremos a mesma decomposição que utilizamos na prova da ida. E chegamos, novamente, à fórmula $det(X) = det(\bar{X})det(\alpha - y^T\bar{X}y)$. Sabemos que $det(X) > 0$ e $det(\bar{X}) > 0$, logo $det(\alpha - y^T\bar{X}y) > 0$, porém, $\alpha - y^T\bar{X}y \in \R$, logo, só tem determinante positivo se for positivo, portanto, novamente pelo Ex. 18, $X \succ 0$.
\end{proof}
\end{homeworkProblem}

\begin{homeworkProblem}
Prove que $\mathrm{int}(\S_+^n) = \S_{++}^n$.

\begin{proof}[Resposta]
Primeiro, vamos provar $\S_{++}^n \subseteq \mathrm{int}(\S_+^n)$. Seja $x \in \S_{++}^n$ e $\epsilon = \max \limits_{\substack{u \in \mathbb{B} \\ {h^Tuh \neq 0}}} |\frac{h^Txh}{h^Tuh}|$.  

Temos que para todo $u \in \mathbb{B}$
$$ h^T(x+\epsilon u)h = h^Txh + \epsilon h^Tuh \text{, }$$
o que nos dá dois casos:

\begin{enumerate}
\item se $h^Tuh \geq 0$, $h^T(x + \epsilon u)h \geq h^Txh > 0$;
\item se $h^Tuh < 0$, $h^T(x + \epsilon u)h = h^Tuh + \epsilon h^Tuh \geq h^Txh - h^Txh = 0$.
\end{enumerate}

Desta forma, em todos os casos possíveis, $h^Txh \geq 0$, logo, $x \in \S_+^n$.  

Agora, vamos provar que $\mathrm{int}\S_+^n \subseteq \S_{++}^n$. Seja $x \in \mathrm{int}(\S_+^n)$. Então existe $\epsilon > 0$ tal que para todo $h \in \R^n$ e $u \in \mathbb{B}$, 
$$ h^T(x + \epsilon u)h \geq 0 \text{, portanto} $$
$$ h^Txh + \epsilon h^Tuh \geq 0 \text{.}$$
Escolha $u = -I/\sqrt(n)$ e qualquer $h \neq 0$.
$$ h^Txh - \frac{\epsilon}{\sqrt{n}} \geq 0 \text{, }$$
$$ h^Txh \geq \frac{\epsilon}{\sqrt{n}} \geq 0 \text{, ou seja}$$
$$ x \in \S_{++}^n \text{.}$$

Com isso, concluímos que $\mathrm{int}(\S_+^n) = \S_{++}^n$.
\end{proof}
\end{homeworkProblem}

\end{document}
