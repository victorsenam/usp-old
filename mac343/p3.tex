%%%%%%%%%%%%%%%%%%%%%%%%%%%%%%%%%%%%%%%%%
% Programming/Coding Assignment
% LaTeX Template
%
% This template has been downloaded from:
% http://www.latextemplates.com
%
% Original author:
% Ted Pavlic (http://www.tedpavlic.com)
%
% Note:
% The \lipsum[#] commands throughout this template generate dummy text
% to fill the template out. These commands should all be removed when 
% writing assignment content.
%
% This template uses a Perl script as an example snippet of code, most other
% languages are also usable. Configure them in the "CODE INCLUSION 
% CONFIGURATION" section.
%
%%%%%%%%%%%%%%%%%%%%%%%%%%%%%%%%%%%%%%%%%

%----------------------------------------------------------------------------------------
%	PACKAGES AND OTHER DOCUMENT CONFIGURATIONS
%----------------------------------------------------------------------------------------

\documentclass{article}
\usepackage[utf8]{inputenc}
\usepackage[brazil]{babel}

\usepackage{amsmath,amsthm,amssymb,amsfonts,xfrac} % Math stuff
\usepackage[noend]{algpseudocode}
\usepackage{algorithmicx}
%\usepackage{enumitem,tikz}
\usepackage{enumerate} % Better enumerates
\usepackage{dsfont}

\usepackage{fancyhdr} % Required for custom headers
\usepackage{lastpage} % Required to determine the last page for the footer
\usepackage{extramarks} % Required for headers and footers
\usepackage[usenames,dvipsnames]{color} % Required for custom colors
\usepackage{graphicx} % Required to insert images
\usepackage{listings} % Required for insertion of code
\usepackage{courier} % Required for the courier font
\usepackage{lipsum} % Used for inserting dummy 'Lorem ipsum' text into the template

% Margins
\topmargin=-0.45in
\evensidemargin=0in
\oddsidemargin=0in
\textwidth=6.5in
\textheight=9.0in
\headsep=0.25in

\linespread{1.1} % Line spacing

% Set up the header and footer
\pagestyle{fancy}
\lhead{\hmwkAuthorName} % Top left header
\chead{\hmwkClass: \hmwkTitle} % Top center head
\rhead{\firstxmark} % Top right header
\lfoot{\lastxmark} % Bottom left footer
\cfoot{} % Bottom center footer
\rfoot{Página\ \thepage\ de\ \protect\pageref{LastPage}} % Bottom right footer
\renewcommand\headrulewidth{0.4pt} % Size of the header rule
\renewcommand\footrulewidth{0.4pt} % Size of the footer rule

\setlength\parindent{0pt} % Removes all indentation from paragraphs

%----------------------------------------------------------------------------------------
%	DOCUMENT STRUCTURE COMMANDS
%	Skip this unless you know what you're doing
%----------------------------------------------------------------------------------------

% Header and footer for when a page split occurs within a problem environment
\newcommand{\enterProblemHeader}[1]{
\nobreak\extramarks{#1}{#1 continua na próxima página\ldots}\nobreak
\nobreak\extramarks{#1 (continua)}{#1 continua na próxima página\ldots}\nobreak
}

% Header and footer for when a page split occurs between problem environments
\newcommand{\exitProblemHeader}[1]{
\nobreak\extramarks{#1 (continua)}{#1 continua na próxima página\ldots}\nobreak
\nobreak\extramarks{#1}{}\nobreak
}

\setcounter{secnumdepth}{0} % Removes default section numbers
\newcounter{homeworkProblemCounter} % Creates a counter to keep track of the number of problems

\newcommand{\homeworkProblemName}{}
\newenvironment{homeworkProblem}[1][\unskip]{ % Sets homework environment with adittional argument for extra naming
    \stepcounter{homeworkProblemCounter} % Increase counter for number of problems
    \renewcommand{\homeworkProblemName}{Problema \arabic{homeworkProblemCounter} #1} % Assign \homeworkProblemName the name of the problem
    \section{\homeworkProblemName} % Make a section in the document with the custom problem count
    \enterProblemHeader{\homeworkProblemName} % Header and footer within the environment
}{
\exitProblemHeader{\homeworkProblemName} % Header and footer after the environment
}

\newcommand{\homeworkSectionName}{}
\newenvironment{homeworkSection}[1]{ % New environment for sections within homework problems, takes 1 argument - the name of the section
\renewcommand{\homeworkSectionName}{#1} % Assign \homeworkSectionName to the name of the section from the environment argument
\subsection{\homeworkSectionName} % Make a subsection with the custom name of the subsection
\enterProblemHeader{\homeworkProblemName\ [\homeworkSectionName]} % Header and footer within the environment
}{
\enterProblemHeader{\homeworkProblemName} % Header and footer after the environment
}

\newenvironment{homeworkProblemAnswer}[0]{ % Defines the problem answer environment
    \begin{proof}[\textbf{Resposta}]
    }{
    \end{proof}
}

%----------------------------------------------------------------------------------------
%	ALGPSEUDOCODE CONFIGURATION
%----------------------------------------------------------------------------------------

\algrenewtext{Function}[2]{\textbf{função} \textsc{#1}$(#2)$}
\algrenewtext{For}[1]{\textbf{para} #1 \textbf{faça}}
\algrenewtext{If}[1]{\textbf{se} #1 \textbf{então}}
\algrenewtext{ElsIf}[1]{\textbf{senão se} #1 \textbf{então}}
\algrenewcommand\Return{\textbf{devolve} }

%----------------------------------------------------------------------------------------
%	THEOREMS AND COUNTERS
%----------------------------------------------------------------------------------------

\numberwithin{equation}{homeworkProblemCounter}

\newtheorem{prop}[equation]{Proposição}

%----------------------------------------------------------------------------------------
%	NAME AND CLASS SECTION
%----------------------------------------------------------------------------------------

\newcommand{\hmwkTitle}{Prova 2} % Assignment title
\newcommand{\hmwkDueDate}{22 de Outubro de 2016} % Due date
\newcommand{\hmwkClass}{MAC0343} % Course/class
\newcommand{\hmwkAuthorName}{Victor Sena Molero - 8941317} % Assignment author

%----------------------------------------------------------------------------------------
%	TITLE PAGE
%----------------------------------------------------------------------------------------

\title{
\vspace{2in}
\textmd{\textbf{\hmwkClass:\ \hmwkTitle}}\\
\normalsize\vspace{0.1in}\small{\hmwkDueDate}\\
\vspace{3in}
}

\author{\textbf{\hmwkAuthorName}}
\date{} % Insert date here if you want it to appear below your name

%----------------------------------------------------------------------------------------
%	TABLE OF CONTENTS
%----------------------------------------------------------------------------------------

%\setcounter{tocdepth}{1} % Uncomment this line if you don't want subsections listed in the ToC


%----------------------------------------------------------------------------------------
%>-- SHORTCUTS
%----------------------------------------------------------------------------------------

\renewcommand{\|}[1]{\mathbb{#1}}
\renewcommand{\"}[1]{\ensuremath{\mathcal{#1}}}

\newcommand{\sse}{\Leftrightarrow}
\newcommand{\se}{\Rightarrow}
\newcommand{\so}{\Leftarrow}

\newcommand{\conv}{\textrm{conv}}
\newcommand{\T}{\textrm{TH}}

%----------------------------------------------------------------------------------------

\begin{document}

%\maketitle
%\newpagea

%\tableofcontents
%\newpage

\begin{homeworkProblem}
Sejam $J \subseteq{[n]}$ e $i \in [n] \setminus J$. Prove que $F_i \circ F_J = F_{J\cup\{i\}}$.
\end{homeworkProblem}

\begin{homeworkProblemAnswer}
Para facilitar a prova, vou criar uma proposição.
\begin{prop} \label{prop_1}
Se $a,b \in [0,1], \lambda \in (0,1)$ e $x \in \{0,1\}$, então, $a = b = x$.
\end{prop}

\begin{proof}
Se $a \neq b$, suponha $a < b$. Se $x = 0$, temos, $a < b \se (1-\lambda)a < (1-\lambda)b \se a < \lambda a + (1 - \lambda)b = x = 0$, logo, $a < 0$, uma contradição. Se $x = 1$, temos, $a < b \se \lambda a < \lambda b \se 1 = x = \lambda a + (1-\lambda) b < b$, logo, $1 < b$, uma contradição. Portanto, $a = b$, mas $x = \lambda a + (1 - \lambda)b = a = b$.
\end{proof}

Seja $P \cup [0,1]^V$ um conjunto convexo. Para todo $J \subseteq [n]$, definimos $B_J := \{x \in \|{R}^n \mid x_j \in \{0,1\} \forall j \in J\}$.

Se $x \in F_i \circ F_J(P)$, $x \in \conv(F_J(P) \cap B_i)$, logo, existem $a,b \in F_J(P) \cap B_i$ e $\lambda \in (0,1)$ tais que $x = \lambda a + (1 - \lambda) b$. Já que $a,b \in F_J(P) \cap B_i$, existem $\check{a}, \hat{a}, \check{b}, \hat{b} \in P \cap B_J$ e $\alpha, \beta \in (0,1)$ tais que $a = \lambda \check{a} + (1 - \lambda) \hat{a}$ e $b = \lambda \check{b} + (1 - \lambda) \hat{b}$.  

Temos $a_i, b_i \in \{0,1\}$, logo, por \ref{prop_1}, $\check{a}_i = \hat{a}_i \in \{0,1\}$ e $\check{b}_i = \hat{b}_i \in \{0,1\}$. Assim, $\check{a}, \hat{a}, \check{b}, \hat{b} \in B_i$, logo, $\check{a}, \hat{a}, \check{b}, \hat{b} \in B_J \cup \{i\}$. Sabemos que $x = \lambda \alpha \check{a} + \lambda (1-\alpha) \hat{a} + (1-\lambda) \beta \check{b} + (1-\lambda) (1-\beta) \hat{b}$, portanto, $x$ é combinação convexa de elementos de $P \cap B_{J\cup\{i\}}$, logo, $x \in \conv(P\cap B_{J\cup\{i\}}) = F_{J\cup\{i\}}.$ Ou seja, $F_i \circ F_J(P) \subseteq F_{J\cup\{i\}}(P)$.  

Por outro lado, $F_{J\cup\{i\}}(P) = \conv(P\cap B_{J\cup\{i\}})$. Porém, $P \cap B_{J\cup\{i\}} = P \cap B_J \cap B_i \subseteq \conv(P\cap B_J)\cap B_i = F_J(P)\cap B_i$, logo, $F_{J \cup \{i\}}(P) \subseteq \conv(F_J(P)\cap B_i) = F_i \circ F_J(P)$.
\end{homeworkProblemAnswer}

\begin{homeworkProblem}
Seja $G$ um grafo. Prove que $\T(G)$ é um canto convexo.
\end{homeworkProblem}

\begin{homeworkProblemAnswer}
Seja $G$ um grafo.
\begin{prop}
$\T(G)$ é compacto.
\end{prop}
\begin{proof}
$2I \in \|{S}_{++}^V$, portanto, existe $\delta > 0$ tal que $2I + \delta \|{B} \in \|{S}_{++}^V$ é aberto. Defina $B := \|{1}\|{1}^T$ e escolha $\varepsilon = ||B||/\delta$. Agora, defina $\tilde{T} := \begin{bmatrix}
\varepsilon & -\|{1}^T \\
-\|{1} & 2I 
\end{bmatrix}$.  

Já que $\varepsilon > 0$, $2I \in \|{S}^V$ e $2I - B/\varepsilon = 2I - \delta B/||B||$, porém, $B/||B|| \in \|{B}$, logo, $I + \delta B \in \|{S}_{++}^V$ e $2I \succ B/\varepsilon$, portanto, pelo Ex. 18, $\tilde{T} \in \|{S}_{++}^V.$  

Definimos $T := \{X \in \|{S}_+^{\{0\}\cup B}\}$. Pelo Ex. 1 da P2, $T$ é compacto. Além disso, seja $\hat{X} := \begin{bmatrix}
1 & x^T \\
x & X
\end{bmatrix} \in \hat{\T}(G)$ com $x \in \|{R}^V$ e $X \in \|{S}^V$. $\langle \tilde{T}, \hat{X} \rangle = 1\varepsilon - 2 \langle \|{1}, x \rangle + \langle 2I, X \rangle = \varepsilon$, portanto, $\hat{\T}(G) \subseteq{T}$ é limitado.  

Sabemos que $\|{S}_+^V$ é fechado. Além disso, podemos re-escrever $\hat{\T}(G) = \{X \in \|{S}^{\{0\}\cup V} \mid X_{i,i} = X_{0,i} \forall i \in V e X_{i,j} = 0 \forall ij \in E\}$. Assim, fica claro que $\hat{\T}(G)$ é intersecção de hiperplanos de $\|{S}^{\{0\}\cup V}$ com $\|{S}_V^{\{0\}\cup V}$ e, portanto, é fechado.  

Agora, basta definir a função, para todo $i \in V$ a função $f_i:X \in \|{S}^{\{0\}\cup V} \mapsto X_{i,i} \|{R}$. Chamamos de $f$ a função definida pela concatenação de $f_i$ para todo $i$. $f$ é contínua e $\textrm{Im}_f(\hat{\T}(G)) = \T(G)$, portanto, $\T(G)$ é compacto.
\end{proof}

\begin{prop}
$\T(G)$ é convexo.
\end{prop}

\begin{prop}
$\T(G)$ é antibloqueador.
\end{prop}

\begin{prop}
$\T(G)$ tem interior não vazio.
\end{prop}

\end{homeworkProblemAnswer}

\end{document}
