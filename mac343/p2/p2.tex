%%%%%%%%%%%%%%%%%%%%%%%%%%%%%%%%%%%%%%%%%
% Programming/Coding Assignment
% LaTeX Template
%
% This template has been downloaded from:
% http://www.latextemplates.com
%
% Original author:
% Ted Pavlic (http://www.tedpavlic.com)
%
% Note:
% The \lipsum[#] commands throughout this template generate dummy text
% to fill the template out. These commands should all be removed when 
% writing assignment content.
%
% This template uses a Perl script as an example snippet of code, most other
% languages are also usable. Configure them in the "CODE INCLUSION 
% CONFIGURATION" section.
%
%%%%%%%%%%%%%%%%%%%%%%%%%%%%%%%%%%%%%%%%%

%----------------------------------------------------------------------------------------
%	PACKAGES AND OTHER DOCUMENT CONFIGURATIONS
%----------------------------------------------------------------------------------------

\documentclass{article}
\usepackage[utf8]{inputenc}
\usepackage[brazil]{babel}

\usepackage{amsmath,amsthm,amssymb,amsfonts,xfrac} % Math stuff
\usepackage[noend]{algpseudocode}
\usepackage{algorithmicx}
%\usepackage{enumitem,tikz}
\usepackage{enumerate} % Better enumerates
\usepackage{dsfont}

\usepackage{fancyhdr} % Required for custom headers
\usepackage{lastpage} % Required to determine the last page for the footer
\usepackage{extramarks} % Required for headers and footers
\usepackage[usenames,dvipsnames]{color} % Required for custom colors
\usepackage{graphicx} % Required to insert images
\usepackage{listings} % Required for insertion of code
\usepackage{courier} % Required for the courier font
\usepackage{lipsum} % Used for inserting dummy 'Lorem ipsum' text into the template

% Margins
\topmargin=-0.45in
\evensidemargin=0in
\oddsidemargin=0in
\textwidth=6.5in
\textheight=9.0in
\headsep=0.25in

\linespread{1.1} % Line spacing

% Set up the header and footer
\pagestyle{fancy}
\lhead{\hmwkAuthorName} % Top left header
\chead{\hmwkClass: \hmwkTitle} % Top center head
\rhead{\firstxmark} % Top right header
\lfoot{\lastxmark} % Bottom left footer
\cfoot{} % Bottom center footer
\rfoot{Página\ \thepage\ de\ \protect\pageref{LastPage}} % Bottom right footer
\renewcommand\headrulewidth{0.4pt} % Size of the header rule
\renewcommand\footrulewidth{0.4pt} % Size of the footer rule

\setlength\parindent{0pt} % Removes all indentation from paragraphs

%----------------------------------------------------------------------------------------
%	DOCUMENT STRUCTURE COMMANDS
%	Skip this unless you know what you're doing
%----------------------------------------------------------------------------------------

% Header and footer for when a page split occurs within a problem environment
\newcommand{\enterProblemHeader}[1]{
\nobreak\extramarks{#1}{#1 continua na próxima página\ldots}\nobreak
\nobreak\extramarks{#1 (continua)}{#1 continua na próxima página\ldots}\nobreak
}

% Header and footer for when a page split occurs between problem environments
\newcommand{\exitProblemHeader}[1]{
\nobreak\extramarks{#1 (continua)}{#1 continua na próxima página\ldots}\nobreak
\nobreak\extramarks{#1}{}\nobreak
}

\setcounter{secnumdepth}{0} % Removes default section numbers
\newcounter{homeworkProblemCounter} % Creates a counter to keep track of the number of problems

\newcommand{\homeworkProblemName}{}
\newenvironment{homeworkProblem}[1][\unskip]{ % Sets homework environment with adittional argument for extra naming
    \stepcounter{homeworkProblemCounter} % Increase counter for number of problems
    \renewcommand{\homeworkProblemName}{Problema \arabic{homeworkProblemCounter} #1} % Assign \homeworkProblemName the name of the problem
    \section{\homeworkProblemName} % Make a section in the document with the custom problem count
    \enterProblemHeader{\homeworkProblemName} % Header and footer within the environment
}{
\exitProblemHeader{\homeworkProblemName} % Header and footer after the environment
}

\newcommand{\homeworkSectionName}{}
\newenvironment{homeworkSection}[1]{ % New environment for sections within homework problems, takes 1 argument - the name of the section
\renewcommand{\homeworkSectionName}{#1} % Assign \homeworkSectionName to the name of the section from the environment argument
\subsection{\homeworkSectionName} % Make a subsection with the custom name of the subsection
\enterProblemHeader{\homeworkProblemName\ [\homeworkSectionName]} % Header and footer within the environment
}{
\enterProblemHeader{\homeworkProblemName} % Header and footer after the environment
}

\newenvironment{homeworkProblemAnswer}[0]{ % Defines the problem answer environment
    \begin{proof}[\textbf{Resposta}]
    }{
    \end{proof}
}

%----------------------------------------------------------------------------------------
%	ALGPSEUDOCODE CONFIGURATION
%----------------------------------------------------------------------------------------

\algrenewtext{Function}[2]{\textbf{função} \textsc{#1}$(#2)$}
\algrenewtext{For}[1]{\textbf{para} #1 \textbf{faça}}
\algrenewtext{If}[1]{\textbf{se} #1 \textbf{então}}
\algrenewtext{ElsIf}[1]{\textbf{senão se} #1 \textbf{então}}
\algrenewcommand\Return{\textbf{devolve} }

%----------------------------------------------------------------------------------------
%	THEOREMS AND COUNTERS
%----------------------------------------------------------------------------------------

\numberwithin{equation}{homeworkProblemCounter}

\newtheorem{prop}[equation]{Proposição}

%----------------------------------------------------------------------------------------
%	NAME AND CLASS SECTION
%----------------------------------------------------------------------------------------

\newcommand{\hmwkTitle}{Prova 2} % Assignment title
\newcommand{\hmwkDueDate}{22 de Outubro de 2016} % Due date
\newcommand{\hmwkClass}{MAC0343} % Course/class
\newcommand{\hmwkAuthorName}{Victor Sena Molero - 8941317} % Assignment author

%----------------------------------------------------------------------------------------
%	TITLE PAGE
%----------------------------------------------------------------------------------------

\title{
\vspace{2in}
\textmd{\textbf{\hmwkClass:\ \hmwkTitle}}\\
\normalsize\vspace{0.1in}\small{\hmwkDueDate}\\
\vspace{3in}
}

\author{\textbf{\hmwkAuthorName}}
\date{} % Insert date here if you want it to appear below your name

%----------------------------------------------------------------------------------------
%	TABLE OF CONTENTS
%----------------------------------------------------------------------------------------

%\setcounter{tocdepth}{1} % Uncomment this line if you don't want subsections listed in the ToC


%----------------------------------------------------------------------------------------
%>-- SHORTCUTS
%----------------------------------------------------------------------------------------

\renewcommand{\|}[1]{\mathbb{#1}}
\renewcommand{\"}[1]{\ensuremath{\mathcal{#1}}}

\newcommand{\sse}{\Leftrightarrow}
\newcommand{\se}{\Rightarrow}
\newcommand{\so}{\Leftarrow}

\newcommand{\eigen}{\lambda^{\downarrow}}
\newcommand{\Tr}{\textrm{Tr}}

%----------------------------------------------------------------------------------------

\begin{document}

%\maketitle
%\newpage

%\tableofcontents
%\newpage

\begin{homeworkProblem}[]
Sejam $\bar{S} \in \|{S}_{++}^n$ e $\alpha \in \|{R}_{++}$. Prove que os conjuntos
$$\{X \in \|{S}_{++}^{n} \mid \langle \bar{S}, X \rangle \leq \alpha \} \text{ e } 
  \{X \in \|{S}_{++}^{n} \mid \langle \bar{S}, X \rangle = \alpha \}$$
são não-vazios, convexos e compactos. Mostre ainda que o interior do primeiro conjunto não é vazio.
\end{homeworkProblem}


\begin{homeworkProblemAnswer}
Vamos chamar o primeiro conjunto de $A$ e o segundo de $B$, isto é
$$A = \{X \in \|{S}_{++}^{n} \mid \langle \bar{S}, X \rangle \leq \alpha \} \text{ e } $$
$$B = \{X \in \|{S}_{++}^{n} \mid \langle \bar{S}, X \rangle = \alpha \} \text{.} $$
Agora, devemos provar algumas propriedades sobre $A$ e $B$.
\begin{prop}
$A$ e $B$ são não-vazios.
\end{prop}
\begin{proof}
Já que $\bar{S} \in \|{S}_{++}^n$, $\bar{S} \neq 0$, logo $\langle \bar{S}, \bar{S} \rangle \neq 0$. Escolhermos $\beta = \dfrac{\alpha}{\langle \bar{S}, \bar{S} \rangle}$. Já que $\alpha > 0$ e $\langle \bar{S}, \bar{S} \rangle > 0$, $\beta > 0$. Agora, escolhemos $\bar{X} = \beta\bar{S}$. Temos que para todo $h \in \|{R}^n$,  
$$ h^T\bar{X}h = h^T\beta\bar{S}h = \beta h^T\bar{S}h > 0 \text{, } $$
portanto, $\bar{X} \in \|{S}_{++}^n \subseteq \|{S}_{+}^n$. Além disso, $\langle \bar{S}, X \rangle = \alpha$. Logo, $\bar{X} \in A$ e $\bar{X} \in B$. Portanto, $A \neq \emptyset$ e $B \neq \emptyset$.
\end{proof}

\begin{prop}
$A$ e $B$ são convexos.
\end{prop}
\begin{proof}
Agora, queremos mostrar que $A$ e $B$ são convexos. Sejam $X,Y \in \|{S}_{+}^n$ quaisquer escolhemos $Z = (X+Y)/2$. Temos que, para todo $h \in \|{R}^n$,

$$h^TZh = h^T(X+Y)h/2 = (h^TXh + h^TYh)/2 > 0 \text{, }$$
então $Z \in \|{S}_{++}^n$, além disso,
$$\langle Z, \bar{S} \rangle = (\langle X, \bar{S} \rangle + \langle Y, \bar{S} \rangle)/2 \text{.}$$

Assim, se existe $\beta \in \|{R}$ tal que $\langle X, \bar{S} \rangle = \langle Y, \bar{S} \rangle = \beta$, então $\langle Z, \bar{S} = \beta$, ou seja, $B$ é convexo. Além disso, se existe $\beta \in \|{R}$ tal que $\langle X, \bar{S} \rangle \leq \beta$ e $\langle Y, \bar{S} \rangle \leq \beta$, então, $\langle Z, \bar{S} \rangle \leq \beta$, ou seja, $A$ é convexo.
\end{proof}

\begin{prop}
$A$ e $B$ são limitados.
\end{prop}
\begin{proof}
Sejam $T \in \|{S}^n \setminus \{0\}$ e $X \in A$. Sabemos que $A$ é limitado se e somente se existe $\theta \in \|{R}_+$ tal que $X + \theta T \notin A$.  

Se $\langle T, \bar{S} > 0 \rangle$, basta escolher $\theta = \dfrac{\alpha - \langle X, \bar{S} \rangle}{\langle T, \bar{S} \rangle} + 1$. Já que $\alpha \geq \langle X, \bar{S} \rangle$, $\theta \geq 1 > 0$, logo, $\theta \in \|{R}_+$ e $\langle X + \theta T, \bar{S} \rangle = \alpha + \langle T, \bar{S} \rangle > \alpha$, logo, $X + \theta T \notin A$.  

Caso contrário, pelo \textbf{Ex. 21}, já que $X \in \|{S}_{++}^n$, $T \notin \|{S}_+^n \setminus {0}$. Logo, existe $h \in \|{R}^n$ tal que
$$ h^TTh < 0 \text{, portanto,} $$
se $\theta = -\dfrac{h^TXh}{h^TTh} + 1$, $\theta \geq 1 > 0$. Logo, $\theta \in \|{R}_+$ e já que
$$ h^T(X+\theta T)h = h^TXh + \theta h^TTh = 0 + h^TTh < 0 \text{, } $$
$X + \theta T \notin \|{S}_{++}^n$, logo, $X + \theta T \notin A$. Assim, mostramos que $A$ é limitado. Já que $B \subseteq A$, $B$ também é limitado.
\end{proof}

\begin{prop}
$A$ e $B$ são fechados.
\end{prop}
\begin{proof}
Sabemos que $\|{S}_{++}^n$ é fechado. $C = \{X \in \|{S} \mid \langle X, \bar{S} \rangle \leq \alpha\}$ é um semiespaço, logo, é fechado. $D = \{X \in \|{S} \mid \langle X, \bar{S} \rangle = \alpha\}$ é um hiperplano, logo, é fechado. $A = \|{S}_{++}^n \cap C$ e $B = \|{S}_{++}^n \cap D$, ou seja, tanto $A$ quanto $B$ são fechados.
\end{proof}

\begin{prop}
$A$ tem interior não vazio.
\end{prop}
\begin{proof}
Seja $\displaystyle X = \frac{\alpha \bar{S}}{2 \langle \bar{S}, \bar{S} \rangle} \in A$. Já que $\|{S}_{++}^n$ é aberto, existe um $\bar{\theta} \in \|{R}_+$ tal que $X + \theta T \in \|{S}_{++}^n$ para todo $T \in \|{B}$ e $\theta$ que respeite $\bar{\theta} \geq \theta \in \|{R}_+$. Escolhemos agora $\theta = \min(\bar{\theta}, \dfrac{\alpha}{4\langle \bar{S}, \bar{S} \rangle})$. Temos que, para todo $T \in \|{B}$,
$$ \langle X + \theta T, \bar{S} \rangle = \langle X + \bar{S} \rangle + \theta \langle T, \bar{S} \rangle = \alpha/2 + \theta \langle T, \bar{S} \rangle \text{,} $$
por Cachy-Schwartz (\textbf{Teo. 37}) e pela definição de $\theta$, respectivamente, temos
$$ \alpha/2 + \theta \langle T, \bar{s} \rangle \leq \alpha/2 + \theta \langle \bar{S}, \bar{S} \rangle \leq \alpha/2 + \alpha/4 \leq \alpha \text{.}$$
Portanto, para todo $T \in \|{B}$, $X + \theta T \in A$, logo, $X$ pertence ao interior de $A$ e o interior de $A$ é não-vazio.
\end{proof}

Com isso, temos que $A$ e $B$ são não-vazios, convexos e compactos e $A$ tem interior não-vazio, como pedido pelo execício.
\end{homeworkProblemAnswer}

\begin{homeworkProblem}
Seja $\"{A}: \|{S}^n \rightarrow \|{R}^m$ uma função linear e $b \in \|{R}^m$. Seja $w \in \|{R}^m$ tal que $w_1 \geq \dots \geq w_n$. Formule o seguinte problema de otimização como um programa semidefinido:
\begin{equation*}
    \begin{array}[!h]{lll}
        \text{Minimizar} & w^T \eigen(X)                                    & \\[2pt]
        \text{sujeito a} & \"{A}(X) = b,                                    & \\[2pt]
                         & X \in \|{S}^n.                                   & \\
    \end{array}
\end{equation*}
\end{homeworkProblem}

\begin{homeworkProblemAnswer}
\begin{equation}\label{eq:prob_ex2}
    \begin{array}[!h]{lll}
        \text{Minimizar} & w^T \eigen(X)                                    & \\[2pt]
        \text{sujeito a} & \"{A}(X) = b,                                    & \\[2pt]
                         & X \in \|{S}^n.                                   & \\
    \end{array}
    =
    \begin{array}[!h]{lll}
        \text{Minimizar} & \mu                                              & \\[2pt]
        \text{sujeito a} & \"{A}(X) = b,                                    & \\[2pt]
                         & X \in \|{S}^n,                                   & \\[2pt]
                         & \mu \in \|{R},                                   & \\[2pt]
                         & w^T \eigen(X) \leq \mu.                          & \\
    \end{array}
\end{equation}

Vamos definir $\gamma \in \|{R}^{0 \oplus [n]}$ e $\hat{w} \in \|{R}^n$ da seguinte maneira:
\begin{equation*}
    \begin{array}[!h]{ll}
        \gamma_k = \sum \limits_{i=1}^k \eigen_i(X), & \forall k \in [n], \\[2pt]
        \gamma_0 = 0,                                & \\[2pt]
        \hat{w}_k = w_k - w_{k+1},                   & \forall k \in [n-1], \\[2pt]
        \hat{w} = w_n.                               & \\
    \end{array}
\end{equation*}

Temos
\begin{multline*}  
    w^T \eigen(X) = 
    \sum \limits_{i=1}^n w_i \eigen_i(X) = 
    \sum \limits_{i=1}^n w_i (\gamma_i(X) - \gamma_{i-1}(X)) = 
    \sum \limits_{i=1}^n w_i \gamma_i(X) - \sum \limits_{i=1}^{n-1} w_i \gamma_{i-1}(X) = \\
    \sum \limits_{i=1}^{n-1} \gamma_i(X)(w_i - w_{i+1}) + \gamma_n(X)w_n =
    \sum \limits_{i=1}^n \gamma_i(X) \hat{w}_i =
    \hat{w}^T \gamma[n].
\end{multline*}

Além disso, já que $\hat{w} \geq 0$, para qualquer $\mu \in \|{R}$,
$$
    \hat{w}^T \gamma[n] \leq \mu \sse 
    \text{existe } \hat{\gamma} \in \|{R}^n \text{ tal que } \gamma[n] \leq \hat{\gamma} \text{ e } \hat{w}^T\hat{\gamma} \leq \mu.
$$

Assim, podemos escrever \eqref{eq:prob_ex2} como 
\begin{equation*}
    \begin{array}[!h]{lll}
        \text{Minimizar} & \mu                                              & \\[2pt]
        \text{sujeito a} & \"{A}(X) = b,                                    & \\[2pt]
                         & X \in \|{S}^n,                                   & \\[2pt]
                         & \mu \in \|{R},                                   & \\[2pt]
                         & \hat{\gamma} \in \|{R}^n,                        & \\[2pt]
                         & \gamma[n] \leq \hat{\gamma},                     & \\[2pt]
                         & w^T \eigen(X) \leq \mu.                          & \\
    \end{array}
\end{equation*}

A restrição $\gamma[n] \leq \hat{\gamma}$ equivale a
$$ \gamma_k = \sum \limits_{i=1}^k \eigen_i(X) \leq \hat{\gamma}_k, \forall k \in [n], $$
pelo $\textbf{Teo. 54}$, para todo $k \in [n]$, 
$$  
    \sum \limits_{i=1}^k \eigen_i(X) \leq \hat{\gamma}_k \sse 
    \exists Y \in \|{S}^n \text{ e } \eta \in \|{R} \text{ e } [\nu - k\eta - \Tr(Y)] \oplus Y \oplus [Y - X + \eta I] \in \|{R}_+ \oplus \|{S}_+^n \oplus \|{S}_+^n,
$$
que contém apenas restrições lineares. Desta maneira, conseguimos formular $\eqref{eq:prob_ex2}$ como um programa semidefinido.

\end{homeworkProblemAnswer}

\begin{homeworkProblem}
Seja $\"{A} : \|{R}^n \oplus \|{R} \rightarrow \|{R}^m$ uma função linear. Sejam $b \in \|{R}^m$ e $c \oplus R^m \in \|{R}^n \oplus \|{R}$. Formule o seguinte problema de otimização como um progama cônico e 2a. ordem:
\begin{equation} \label{eq:prob_ex3}
    \begin{array}[!h]{lll}
        \text{Minimizar} & \langle c \oplus \delta, x \oplus \mu \rangle    & \\[2pt]
        \text{sujeito a} & \"{A}(x \oplus \mu) = b,                          & \\[2pt]
                         & ||x||^2 \leq \mu,                                & \\[2pt]
                         & x \oplus \mu \in \|{R}^n \oplus \|{R}.           & \\
    \end{array}
\end{equation}
\end{homeworkProblem}

\begin{homeworkProblemAnswer}
Sabemos que, dados $x \in \|{R}^n$ e $\mu \in \|{R}$,
\begin{multline*}
    ||x||^2 \leq \mu \sse 
    ||x||^2 \leq 1\mu \sse 
    ||x||^2 \leq \left(\dfrac{\mu+1}{2}\right)^2 - \left(\dfrac{\mu-1}{2}\right)^2 \sse 
    ||x||^2  + \left(\dfrac{\mu-1}{2}\right)^2 \leq \left(\dfrac{\mu+1}{2}\right)^2 \sse \\
    ||x \oplus \dfrac{\mu-1}{2}||^2 \leq \left(\dfrac{\mu+1}{2}\right)^2 \sse 
    ||x \oplus \dfrac{\mu-1}{2}|| \leq \dfrac{\mu+1}{2} \sse
    x \oplus \dfrac{\mu-1}{2} \oplus \dfrac{\mu+1}{2} \in \hat{\|{L}}^{n+1}.
\end{multline*}

Com isso, podemos escrever $\eqref{eq:prob_ex3}$ como
\begin{equation*} 
    \begin{array}[!h]{lll}
        \text{Minimizar} & \langle c \oplus \delta, x \oplus \mu \rangle    & \\[2pt]
        \text{sujeito a} & \"{A}(x \oplus \mu) = b,                          & \\[2pt]
                         & x \oplus \left( \frac{\mu-1}{2} \right) \oplus \left( \frac{\mu+1}{2} \right) \in \hat{\|{L}}^{n+1}, & \\[2pt]
                         & x \oplus \mu \in \|{R}^n \oplus \|{R},           & \\
    \end{array}
\end{equation*}
que é um programa cônico de 2a. ordem.
\end{homeworkProblemAnswer}

\begin{homeworkProblem}
Considere o programa semidefinido $\max\{ \langle C, X \rangle : \"{A}(X) = b, X \in \|{S}_+^n \}$, onde $\"{A} : \|{S}^n \rightarrow \|{R}^m$ é uma função linear, $b \in \|{R}^m$ e $C \in \|{S}^n$. Sponha que tanto esse programa como o seu dual possuem pontos de Slater. Prove que, se a região viável do primal é limitada, então a região viável do dual é ilimitada.
\end{homeworkProblem}

\begin{homeworkProblemAnswer}
Sem resposta.
\end{homeworkProblemAnswer}

\begin{homeworkProblem}
Sejam $\bar{a}_1, \dots, \bar{a}_m \in \|{R}^n$ e $\epsilon \in \|{R}_{++}^n$. Sejam $b \in \|{R}^m$ e $c \in \|{R}^n$. Formule o seguinte problema de otimização como um programa cônico de 2a. ordem:
\begin{equation} \label{eq:prob_ex5}
    \begin{array}[!h]{lll}
        \text{Maximizar} & c^Tx                     & \\[2pt]
        \text{sujeito a} & a_i^Tx \leq b_i,         & \forall i \in [m], \forall a_i \in \bar{a}_i + \varepsilon_i \|{B}, \\[2pt]
                         & x \in \|{R}^n.           & \\
    \end{array}
\end{equation}
\end{homeworkProblem}

\begin{homeworkProblemAnswer}
Temos, para todo $i \in [n]$, 
\begin{multline} \label{eq:leq_ex5}
    a_i^Tx \leq b_i \forall a_i \in \bar{a}_i + \varepsilon_i \|{B} \sse 
    \max \limits_{a_i \in \bar{a}_i + \varepsilon_i\|{B}}(a_i^Tx) \leq b_i \sse
    \max \limits_{u \in \|{B}}(\bar{a}_i + \varepsilon_iu)^Tx) \leq b_i \sse \\
    \bar{a}_i + \varepsilon_i \max \limits_{u \in \|{B}}(u^Tx) \leq b_i \sse
    \bar{a}_i + \varepsilon_i \max \limits_{u \in ||x||\|{B}}(u^Tx)/||x|| \leq b_i.
\end{multline}
Por Cauchy-Schwartz, sabemos
$$ \max \limits_{u \in \|{R}^n}(u^Tx) = x^Tx, $$
logo,
$$ \max \limits_{u \in ||x||\|{B}}(u^Tx) = x^Tx,$$
portanto, \eqref{eq:leq_ex5} vale se e somente se
$$ 
    \bar{a}_i^Tx + \varepsilon_i \dfrac{x^Tx}{||x||} = \bar{a}_i^Tx + \varepsilon_i||x|| \leq b_i \sse
    ||x|| \leq (b_i - \bar{a}_i^Tx)/\varepsilon_i \sse
    x \oplus \dfrac{b_i - \bar{a}_i^Tx}{\varepsilon_i} \in \hat{\|{L}}^n.
$$
Ou seja, podemos escrever \eqref{eq:prob_ex5} como 
\begin{equation*}
    \begin{array}[!h]{lll}
        \text{Maximizar} & c^Tx                                                                      & \\[2pt]
        \text{sujeito a} & x \oplus \dfrac{b_i - \bar{a}_i^Tx}{\varepsilon_i} \in \hat{\|{L}}^n,     & \forall i \in [m], \\[2pt]
                         & x \in \|{R}^n.                                                              & \\
    \end{array}
\end{equation*}
\end{homeworkProblemAnswer}

\begin{homeworkProblem}
Considere o programa semidefinido
\begin{equation} \tag{P} \label{eq:prob_ex6}
    \begin{array}[!h]{lll}
        \text{Minimizar} & \langle C, X \rangle                                                      & \\[2pt]
        \text{sujeito a} & \langle e_1e_1^T, X \rangle = 1,                                          & \\[2pt]
                         & X \in \|{S}_+^2,                                                          & \\
    \end{array}
\end{equation}
onde $C := e_1e_2^T + e_2e_1^T$. Mostre que

\begin{enumerate}[(i)]

\item \eqref{eq:prob_ex6} é ilimitado,
\begin{homeworkProblemAnswer}
Seja $\alpha \in \|{R}^n$ e $\bar{X} := e_1e_1^T + \alpha C + \alpha e_2e_2^T$. Temos
$$\langle \bar{X}, e_1e_1^T \rangle = 1$$
e, pela \textbf{Prop. 23}, 
$$\bar{X} \succeq 0 \sse \alpha \alpha \leq \alpha^2,$$
ou seja, $\bar{S} \succeq 0$. Logo, $\bar{X}$ é viável em \eqref{eq:prob_ex6} e o seu valor objetivo é
$$ \langle C, X \rangle = 2 \alpha, $$
que varia livremente com $\alpha$, ou seja, \eqref{eq:prob_ex6} é ilimitado.
\end{homeworkProblemAnswer}

\item o dual de \eqref{eq:prob_ex6} é inviável e
\begin{homeworkProblemAnswer}
O dual de \eqref{eq:prob_ex6} é
\begin{equation} \tag{D} \label{eq:dual_ex6}
    \begin{array}[!h]{lll}
        \text{Maximizar} & y                                                                         & \\[2pt]
        \text{sujeito a} & y \in \|{R},                                                              & \\[2pt]
                         & yC \in \|{S}_+^2,                                                              & \\
    \end{array}
\end{equation}
pois se chamarmos $\"{A}(X) := \langle C, X \rangle$, temos $\"{A} : \|{S}^n \rightarrow \|{R}$, portanto, a variável dual $y$ deve pertencer a $\|{R}$ e teremos também que a tranformação dual de $\"{A}$ é $\"{A}^*(y) = yC$.  

Porém, para todo $y \in \|{R}$, $\det(yC[0]) = 0$, já que $yC[0] = 0$, portanto, pelo \textbf{Teo. 24}, $yC[0] \notin \|{S}_+^2$, assim, não existe $y$ que respeite às restricões de \eqref{eq:dual_ex6} e este é inviável.
\end{homeworkProblemAnswer}

\item para qualquer solução viável $\bar{X}$ de \eqref{eq:prob_ex6}, não existe $D \in \|{S}^2$ tal que $\bar{X} + \alpha D$ é viável para todo $\alpha \in \|{R}_+$ e $\langle C, D \rangle < 0$.

\begin{homeworkProblemAnswer}
Se $\langle C, D \rangle < 0$, já que $C \succeq 0$, $D \notin \|{S}_+^2$, pelo \textbf{Teo. 20}. Logo, existe $h \in \|{R}^2$ tal que $h^TDh < 0$, portanto, se $\alpha = -\dfrac{h^T\bar{X}h}{h^TDh} + 1$, teremos $\alpha \in \|{R}_+$ e também teremos
$$ h^T(\bar{X} + \alpha D)h = h^T\bar{x}h + \alpha h^TDh = h^T\bar{X}h - h^T\bar{X}h + h^TDh < 0. $$
Portanto, $\bar{X} + \alpha D \notin \|{S}_+^2$, logo, $\bar{X} + \alpha D$ é inviável. Ou seja, para todo $D$ conseguimos um $\alpha$ que torna $\bar{X} + \alpha D$ inviável em \eqref{eq:prob_ex6}.
\end{homeworkProblemAnswer}

\end{enumerate}
\end{homeworkProblem}

\begin{homeworkProblem}
Sejam $Q_0, \dots, Q_m \in \|{S}_+^n$, $c_0, \dots, c_m \in \|{R}^n$ e $b \in \|{R}^m$. Formule o seguinte problema de otimização como um programa semidefinido:
\begin{equation} \label{eq:prob_ex7}
    \begin{array}[!h]{lll}
        \text{Minimizar} & \frac{1}{2}x^TQ_0x + c_0^Tx                                               & \\[2pt]
        \text{sujeito a} & \frac{1}{2}x^TQ_ix + c_i^Tx + b_i \leq 0,                                 & \forall i \in [m] \\[2pt]
                         & x \in \|{R}^n.                                                            & \\
    \end{array}
\end{equation}

\begin{homeworkProblemAnswer}
Para todo $i \in [m]$ e $x \in \|{R}^n$,
$$ 
    \frac{1}{2}x^TQ_ix + c_i^Tx + b_i \leq 0 \sse 
    \text{existe } \alpha \in \|{R} \text{ tal que } x^TQ_ix \leq 2\alpha \text{ e } \alpha + c_i^Tx + b_i \leq 0,
$$
pelo \textbf{Ex. 18}, isso vale se e somente se
\begin{equation*}
    \text{existe } \alpha \in \|{R} \text{ tal que }
    \begin{bmatrix}
    I                     & Q_i^{\sfrac{1}{2}}x \\
    (Q_i^{\sfrac{1}{2}}x)^T & 2\alpha
    \end{bmatrix}
    \succeq 0 \text{ e } \alpha + c_i^Tx + b_i \leq 0.
\end{equation*}
Logo, o programa \eqref{eq:prob_ex7} pode ser formulado como
\begin{equation*}
    \begin{array}[!h]{lll}
        \text{Minimizar} & \alpha_0 + c_0^Tx                                                         & \\[2pt]
        \text{sujeito a} & \begin{bmatrix}
                           I                     & Q_i^{\sfrac{1}{2}}x \\
                           (Q_i^{\sfrac{1}{2}}x)^T & 2\alpha_i
                           \end{bmatrix} \succeq 0,                                                  & \forall i \in 0 \oplus [m], \\[2pt]
                         & \alpha_i + c_i^Tx + b_i \leq 0                                            & \forall i \in [m], \\[2pt]
                         & \alpha \in \|{R} \oplus \|{R}^m,                                          & \\[2pt]
                         & x \in \|{R}^n.                                                            & \\
    \end{array}
\end{equation*}

\end{homeworkProblemAnswer}

\end{homeworkProblem}

\end{document}
