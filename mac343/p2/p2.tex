%%%%%%%%%%%%%%%%%%%%%%%%%%%%%%%%%%%%%%%%%
% Programming/Coding Assignment
% LaTeX Template
%
% This template has been downloaded from:
% http://www.latextemplates.com
%
% Original author:
% Ted Pavlic (http://www.tedpavlic.com)
%
% Note:
% The \lipsum[#] commands throughout this template generate dummy text
% to fill the template out. These commands should all be removed when 
% writing assignment content.
%
% This template uses a Perl script as an example snippet of code, most other
% languages are also usable. Configure them in the "CODE INCLUSION 
% CONFIGURATION" section.
%
%%%%%%%%%%%%%%%%%%%%%%%%%%%%%%%%%%%%%%%%%

%----------------------------------------------------------------------------------------
%	PACKAGES AND OTHER DOCUMENT CONFIGURATIONS
%----------------------------------------------------------------------------------------

\documentclass{article}
\usepackage[utf8]{inputenc}
\usepackage[brazil]{babel}

\usepackage{amsmath,amsthm,amssymb,amsfonts,xfrac} % Math stuff
\usepackage[noend]{algpseudocode}
\usepackage{algorithmicx}
%\usepackage{enumitem,tikz}
\usepackage{enumerate} % Better enumerates
\usepackage{dsfont}

\usepackage{fancyhdr} % Required for custom headers
\usepackage{lastpage} % Required to determine the last page for the footer
\usepackage{extramarks} % Required for headers and footers
\usepackage[usenames,dvipsnames]{color} % Required for custom colors
\usepackage{graphicx} % Required to insert images
\usepackage{listings} % Required for insertion of code
\usepackage{courier} % Required for the courier font
\usepackage{lipsum} % Used for inserting dummy 'Lorem ipsum' text into the template

% Margins
\topmargin=-0.45in
\evensidemargin=0in
\oddsidemargin=0in
\textwidth=6.5in
\textheight=9.0in
\headsep=0.25in

\linespread{1.1} % Line spacing

% Set up the header and footer
\pagestyle{fancy}
\lhead{\hmwkAuthorName} % Top left header
\chead{\hmwkClass: \hmwkTitle} % Top center head
\rhead{\firstxmark} % Top right header
\lfoot{\lastxmark} % Bottom left footer
\cfoot{} % Bottom center footer
\rfoot{Página\ \thepage\ de\ \protect\pageref{LastPage}} % Bottom right footer
\renewcommand\headrulewidth{0.4pt} % Size of the header rule
\renewcommand\footrulewidth{0.4pt} % Size of the footer rule

\setlength\parindent{0pt} % Removes all indentation from paragraphs

%----------------------------------------------------------------------------------------
%	DOCUMENT STRUCTURE COMMANDS
%	Skip this unless you know what you're doing
%----------------------------------------------------------------------------------------

% Header and footer for when a page split occurs within a problem environment
\newcommand{\enterProblemHeader}[1]{
\nobreak\extramarks{#1}{#1 continua na próxima página\ldots}\nobreak
\nobreak\extramarks{#1 (continua)}{#1 continua na próxima página\ldots}\nobreak
}

% Header and footer for when a page split occurs between problem environments
\newcommand{\exitProblemHeader}[1]{
\nobreak\extramarks{#1 (continua)}{#1 continua na próxima página\ldots}\nobreak
\nobreak\extramarks{#1}{}\nobreak
}

\setcounter{secnumdepth}{0} % Removes default section numbers
\newcounter{homeworkProblemCounter} % Creates a counter to keep track of the number of problems

\newcommand{\homeworkProblemName}{}
\newenvironment{homeworkProblem}[1][\unskip]{ % Sets homework environment with adittional argument for extra naming
    \stepcounter{homeworkProblemCounter} % Increase counter for number of problems
    \renewcommand{\homeworkProblemName}{Problema \arabic{homeworkProblemCounter} #1} % Assign \homeworkProblemName the name of the problem
    \section{\homeworkProblemName} % Make a section in the document with the custom problem count
    \enterProblemHeader{\homeworkProblemName} % Header and footer within the environment
}{
\exitProblemHeader{\homeworkProblemName} % Header and footer after the environment
}

\newcommand{\homeworkSectionName}{}
\newenvironment{homeworkSection}[1]{ % New environment for sections within homework problems, takes 1 argument - the name of the section
\renewcommand{\homeworkSectionName}{#1} % Assign \homeworkSectionName to the name of the section from the environment argument
\subsection{\homeworkSectionName} % Make a subsection with the custom name of the subsection
\enterProblemHeader{\homeworkProblemName\ [\homeworkSectionName]} % Header and footer within the environment
}{
\enterProblemHeader{\homeworkProblemName} % Header and footer after the environment
}

\newenvironment{homeworkProblemAnswer}[0]{ % Defines the problem answer environment
    \begin{proof}[\textbf{Resposta}]
    }{
    \end{proof}
}

%----------------------------------------------------------------------------------------
%	ALGPSEUDOCODE CONFIGURATION
%----------------------------------------------------------------------------------------

\algrenewtext{Function}[2]{\textbf{função} \textsc{#1}$(#2)$}
\algrenewtext{For}[1]{\textbf{para} #1 \textbf{faça}}
\algrenewtext{If}[1]{\textbf{se} #1 \textbf{então}}
\algrenewtext{ElsIf}[1]{\textbf{senão se} #1 \textbf{então}}
\algrenewcommand\Return{\textbf{devolve} }

%----------------------------------------------------------------------------------------
%	THEOREMS AND COUNTERS
%----------------------------------------------------------------------------------------

\numberwithin{equation}{homeworkProblemCounter}

\newtheorem{prop}[equation]{Proposição}

%----------------------------------------------------------------------------------------
%	NAME AND CLASS SECTION
%----------------------------------------------------------------------------------------

\newcommand{\hmwkTitle}{Prova 2} % Assignment title
\newcommand{\hmwkDueDate}{22 de Outubro de 2016} % Due date
\newcommand{\hmwkClass}{MAC0343} % Course/class
\newcommand{\hmwkAuthorName}{Victor Sena Molero - 8941317} % Assignment author

%----------------------------------------------------------------------------------------
%	TITLE PAGE
%----------------------------------------------------------------------------------------

\title{
\vspace{2in}
\textmd{\textbf{\hmwkClass:\ \hmwkTitle}}\\
\normalsize\vspace{0.1in}\small{\hmwkDueDate}\\
\vspace{3in}
}

\author{\textbf{\hmwkAuthorName}}
\date{} % Insert date here if you want it to appear below your name

%----------------------------------------------------------------------------------------
%	TABLE OF CONTENTS
%----------------------------------------------------------------------------------------

%\setcounter{tocdepth}{1} % Uncomment this line if you don't want subsections listed in the ToC


%----------------------------------------------------------------------------------------
%>-- SHORTCUTS
%----------------------------------------------------------------------------------------

\renewcommand{\|}[1]{\mathbb{#1}}

\newcommand{\sse}{\Longleftrightarrow}
\newcommand{\se}{\Longrightarrow}
\newcommand{\so}{\Longleftarrow}

%----------------------------------------------------------------------------------------

\begin{document}

%\maketitle
%\newpage

%\tableofcontents
%\newpage

\begin{homeworkProblem}[]
Sejam $\bar{S} \in \|{S}_{++}^n$ e $\alpha \in \|{R}_{++}$. Prove que os conjuntos
$$\{X \in \|{S}_{++}^{n} \mid \langle \bar{S}, X \rangle \leq \alpha \} \text{ e } 
  \{X \in \|{S}_{++}^{n} \mid \langle \bar{S}, X \rangle = \alpha \}$$
são não-vazios, convexos e compactos. Mostre ainda que o interior do primeiro conjunto não é vazio.
\end{homeworkProblem}


\begin{homeworkProblemAnswer}
Vamos chamar o primeiro conjunto de $A$ e o segundo de $B$, isto é
$$A = \{X \in \|{S}_{++}^{n} \mid \langle \bar{S}, X \rangle \leq \alpha \} \text{ e } $$
$$B = \{X \in \|{S}_{++}^{n} \mid \langle \bar{S}, X \rangle = \alpha \} \text{.} $$
Agora, devemos provar algumas propriedades sobre $A$ e $B$.
\begin{prop}
$A$ e $B$ são não-vazios.
\end{prop}
\begin{proof}
Já que $\bar{S} \in \|{S}_{++}^n$, $\bar{S} \neq 0$, logo $\langle \bar{S}, \bar{S} \rangle \neq 0$. Escolhermos $\beta = \dfrac{\alpha}{\langle \bar{S}, \bar{S} \rangle}$. Já que $\alpha > 0$ e $\langle \bar{S}, \bar{S} \rangle > 0$, $\beta > 0$. Agora, escolhemos $\bar{X} = \beta\bar{S}$. Temos que para todo $h \in \|{R}^n$,  
$$ h^T\bar{X}h = h^T\beta\bar{S}h = \beta h^T\bar{S}h > 0 \text{, } $$
portanto, $\bar{X} \in \|{S}_{++}^n \subseteq \|{S}_{+}^n$. Além disso, $\langle \bar{S}, X \rangle = \alpha$. Logo, $\bar{X} \in A$ e $\bar{X} \in B$. Portanto, $A \neq \emptyset$ e $B \neq \emptyset$.
\end{proof}

\begin{prop}
$A$ e $B$ são convexos.
\end{prop}
\begin{proof}
Agora, queremos mostrar que $A$ e $B$ são convexos. Sejam $X,Y \in \|{S}_{+}^n$ quaisquer escolhemos $Z = (X+Y)/2$. Temos que, para todo $h \in \|{R}^n$,

$$h^TZh = h^T(X+Y)h/2 = (h^TXh + h^TYh)/2 > 0 \text{, }$$
então $Z \in \|{S}_{++}^n$, além disso,
$$\langle Z, \bar{S} \rangle = (\langle X, \bar{S} \rangle + \langle Y, \bar{S} \rangle)/2 \text{.}$$

Assim, se existe $\beta \in \|{R}$ tal que $\langle X, \bar{S} \rangle = \langle Y, \bar{S} \rangle = \beta$, então $\langle Z, \bar{S} = \beta$, ou seja, $B$ é convexo. Além disso, se existe $\beta \in \|{R}$ tal que $\langle X, \bar{S} \rangle \leq \beta$ e $\langle Y, \bar{S} \rangle \leq \beta$, então, $\langle Z, \bar{S} \rangle \leq \beta$, ou seja, $A$ é convexo.
\end{proof}

\begin{prop}
$A$ e $B$ são limitados.
\end{prop}
\begin{proof}
Sejam $T \in \|{S}^n \setminus \{0\}$ e $X \in A$. Sabemos que $A$ é limitado se e somente se existe $\theta \in \|{R}_+$ tal que $X + \theta T \notin A$.  

Se $\langle T, \bar{S} > 0 \rangle$, basta escolher $\theta = \dfrac{\alpha - \langle X, \bar{S} \rangle}{\langle T, \bar{S} \rangle} + 1$. Já que $\alpha \geq \langle X, \bar{S} \rangle$, $\theta \geq 1 > 0$, logo, $\theta \in \|{R}_+$ e $\langle X + \theta T, \bar{S} \rangle = \alpha + \langle T, \bar{S} \rangle > \alpha$, logo, $X + \theta T \notin A$.  

Caso contrário, pelo \textbf{Ex. 21}, já que $X \in \|{S}_{++}^n$, $T \notin \|{S}_+^n \setminus {0}$. Logo, existe $h \in \|{R}^n$ tal que
$$ h^TTh < 0 \text{, portanto,} $$
se $\theta = -\dfrac{h^TXh}{h^TTh} + 1$, $\theta \geq 1 > 0$. Logo, $\theta \in \|{R}_+$ e já que
$$ h^T(X+\theta T)h = h^TXh + \theta h^TTh = 0 + h^TTh < 0 \text{, } $$
$X + \theta T \notin \|{S}_{++}^n$, logo, $X + \theta T \notin A$. Assim, mostramos que $A$ é limitado. Já que $B \subseteq A$, $B$ também é limitado.
\end{proof}

\begin{prop}\label{prop_abfechados}
$A$ e $B$ são fechados.
\end{prop}
\begin{proof}
Sabemos que $\|{S}_{++}^n$ é fechado. $C = \{X \in \|{S} \mid \langle X, \bar{S} \rangle \leq \alpha\}$ é um semiespaço, logo, é fechado. $D = \{X \in \|{S} \mid \langle X, \bar{S} \rangle = \alpha\}$ é um hiperplano, logo, é fechado. $A = \|{S}_{++}^n \cap C$ e $B = \|{S}_{++}^n \cap D$, ou seja, tanto $A$ quanto $B$ são fechados.
\end{proof}

\begin{prop}
$A$ tem interior não vazio.
\end{prop}
\begin{proof}
\end{proof}

Com isso, temos que $A$ e $B$ são não-vazios, convexos e compactos e $A$ tem interior não-vazio. 
\end{homeworkProblemAnswer}

\end{document}
