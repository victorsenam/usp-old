%%%%%%%%%%%%%%%%%%%%%%%%%%%%%%%%%%%%%%%%%
% Arsclassica Article
% LaTeX Template
% Version 1.1 (10/6/14)
%
% This template has been downloaded from:
% http://www.LaTeXTemplates.com
%
% Original author:
% Lorenzo Pantieri (http://www.lorenzopantieri.net) with extensive modifications by:
% Vel (vel@latextemplates.com)
%
% License:
% CC BY-NC-SA 3.0 (http://creativecommons.org/licenses/by-nc-sa/3.0/)
%
%%%%%%%%%%%%%%%%%%%%%%%%%%%%%%%%%%%%%%%%%

%----------------------------------------------------------------------------------------
%	PACKAGES AND OTHER DOCUMENT CONFIGURATIONS
%----------------------------------------------------------------------------------------

\documentclass[
10pt, % Main document font size
a4paper, % Paper type, use 'letterpaper' for US Letter paper
oneside, % One page layout (no page indentation)
%twoside, % Two page layout (page indentation for binding and different headers)
headinclude,footinclude, % Extra spacing for the header and footer
BCOR5mm, % Binding correction
]{scrartcl}

\input{structure.tex} % Include the structure.tex file which specified the document structure and layout

\hyphenation{Fortran hy-phen-ation} % Specify custom hyphenation points in words with dashes where you would like hyphenation to occur, or alternatively, don't put any dashes in a word to stop hyphenation altogether

%----------------------------------------------------------------------------------------
%	TITLE AND AUTHOR(S)
%----------------------------------------------------------------------------------------

\title{\normalfont\spacedallcaps{Decomposição QR}} % The article title
\subtitle{\normalfont\spacedallcaps{EP3 - MAC0300}}

\author{\spacedlowsmallcaps{Victor Sena Molero (8941317)}} % The article author(s) - author affiliations need to be specified in the AUTHOR AFFILIATIONS block

%----------------------------------------------------------------------------------------

\begin{document}

%----------------------------------------------------------------------------------------
%	HEADERS
%----------------------------------------------------------------------------------------

\renewcommand{\sectionmark}[1]{\markright{\spacedlowsmallcaps{#1}}} % The header for all pages (oneside) or for even pages (twoside)
%\renewcommand{\subsectionmark}[1]{\markright{\thesubsection~#1}} % Uncomment when using the twoside option - this modifies the header on odd pages
\lehead{\mbox{\llap{\small\thepage\kern1em\color{halfgray} \vline}\color{halfgray}\hspace{0.5em}\rightmark\hfil}} % The header style

\pagestyle{scrheadings} % Enable the headers specified in this block

%----------------------------------------------------------------------------------------
%	TABLE OF CONTENTS & LISTS OF FIGURES AND TABLES
%----------------------------------------------------------------------------------------

\maketitle % Print the title/author/date block

\setcounter{tocdepth}{2} % Set the depth of the table of contents to show sections and subsections only

\tableofcontents % Print the table of contents

%\listoffigures % Print the list of figures
%\listoftables % Print the list of tables

%----------------------------------------------------------------------------------------
%	ABSTRACT
%----------------------------------------------------------------------------------------

\section*{Abstract} % This section will not appear in the table of contents due to the star (\section*)

Esse é o relatório sobre o terceiro EP de MAC0300, que tem como objetivo a implementação da Decomposição QR e a resolução do problema dos quadrados mínimos. Os dois conceitos serão revisados brevemente e os principais problemas sobre a implementação serão discutidos com mais cuidado.

%----------------------------------------------------------------------------------------

\newpage % Start the article content on the second page, remove this if you have a longer abstract that goes onto the second page

\section{Introdução}
Além das explicações sobre os tópicos abordados naturalmente pelo EP, este relatório justifica as decisões feitas durante a implementação do algoritmo e mostra como foram criados os métodos para geração de testes para o programa.

\section{O problema dos Quadrados Mínimos}
Com $n \leq m$, ada uma matriz $A \in \mathbb{R}^{n \times m}$ e um vetor $b \in \mathbb{R}^{m}$ encontrar $x \in \mathbb{R}^{n}$ que minimize $||b-Ax||_2$. \\
Isto é equivalente a dizer que dado um sistema sobredeterminado, de $m$ equações com $n$ variáveis onde $n \leq m$, deseja-se encontrar os valores das $n$ variáveis que minimizem o erro entre aplicar as equações a essas variáveis e os resultados dados pelo problema. \\
Uma aplicação clara disso é, dada uma base de polinômios de dimensão $n$, interpolar o polinômio que melhor aproxima alguns pontos dados da função. \\
O problema pode ter uma única ou infinitas soluções. Dependendo da singularidade de $A$. Nosso objetivo será encontrar uma.

\subsection{Decomposição QR}
A decomposição QR encontra, para uma matriz $A \in \mathbb{R}^{n \times m}$, duas matrizes, uma $Q \in \mathbb{R}^{n \times m}$ e uma $R \in \mathbb{R}^{m \times m}$ tais que $A = QR$, $Q$ é ortogonal e $R$ é triangular superior. \\
Para isso, basta escolher inteligentemente uma série de refletores (ou rotatores) $Q^T_i$ na matriz $A$ de forma a transformá-la em uma $R$ triangular superior. Assim, teremos que, se $Q^T$ é o produto de todas as $Q^T_i$ aplicadas, $Q^TA = R$. Já que $Q^T$ é produto de refletores, que são ortogonais, é ortogonal, logo $Q^{-T} = Q$ e $A = QR$, assim obtemos a decomposição QR. \\

\subsection{Usando a Decomposição}
Para resolver o problema dos quadrados mínimos usando QR lembramos que devemos achar $x$ tal que $b-Ax$ tem norma 2 mínima. Porém, se premultiplicarmos a expressão por $Q^T$ teremos: \\
$$ Q^T(b-Ax) = Q^Tb - Q^TAx = Q^Tb - Q^TQRx = Q^Tb - Rx$$ \\
Nós conseguimos encontrar $x$ tal que $Q^Tb = Rx$ com substituição para trás, já que $R$ é triangular superior, assim, nós temos uma solução para um problema dos quadrados mínimos de $\hat{A}$ e $\hat{b}$ onde $\hat{A} = Q^TA$ e $\hat{b} = Q^Tb$. Mas já que $Q$ é ortogonal, $||b-Ax||_2 = ||Q^T(b-Ax)||_2$. Ou seja, ao resolver este problema, resolvemos nosso problema inicial, já que o $x$ que minimiza um também minimiza o outro.

\section{Implementação e problemas}
Em sala de aula foram discutidas algumas dúvidas quanto à implementação e sobre práticas que minimizariam a quantidade de operações ou espaço ou que maximizariam a precisão da solução. O fato é que se $A$ tem sempre posto completo, só existe uma opção a ser feita quando a memória ou tempo. Já se $A$ tem posto incompleto, devemos aplicar o pivoteamento de colunas e existe mais uma forma de implementar isso e de calcular qual coluna deve ser pivotada. Vamos discutir um problema por vez.

\subsection{Aplicando a Reflexão}
É comentado no livro o fato de que existem várias maneiras de aplicar a reflexão, mas o custo de aplicar muda drásticamente. Além disso, se formos aplicar a reflexão orientada a linhas, devemos usar um vetor auxiliar. \\
No meu EP, eu aproveitei o espaço ainda livre no vetor $\gamma$ para realizar os calculos necessários. O vetor em questão, no passo $k$, tem $k$ posições calculadas (incluindo a que é calculada no passo atual). Já que o vetor é inicializado com $m$ espaços, ele tem, no passo $k$, $m-k$ espaços livres. Esta quantidade é exatamente a usada pelo vetor auxiliar mencionado no livro.
 
\subsection{Escalamento}
Para evitar overflows nos cálculos das normas das colunas, é necessário normaliza-las de alguma maneira. Existem alguns jeitos de fazer isso e o livro menciona dois. \\
Inicialmente, falando de problemas com posto completo, o livro fala sobre dividir cada coluna pelo seu máximo elemento, evitando overflow. Depois, para resolver problemas de posto incompleto, é comentado o método de dividir, à priori, a matriz inteira pelo maior elemento da matriz. \\
Eu resolvi utilizar o segundo método. Ele tem vantagens e desvantagens. Além de me parecer mais elegante do que as maneiras que consegui pensar em implementar o outro método, gasta menos memória do que os métodos que eu pensei e os que foram discutidos em aula. \\
Dividir a matriz toda por um só escalar funciona pois minimizar $b-A\alpha * (1/\alpha) * x$ é minimizar $b-Ax$, ou seja, se resolvermos um problema onde dividimos $A$ por $\alpha$, a solução do problema original será o vetor $x$ obtido no modificado com todos os seus elementos multiplicados por $\alpha$.

\section{Testes}
Os testes estão todos na pasta \textit{codigo/exemplos}. \\
Como no EP passado, eu criei geradores que estão em \textit{codigo/exemplos/geradores}. Por enquanto não consegui criar um script para rodar todos, mas acho que vai dar tempo e eu faço (caso contrário, este é um pedido de desculpas). \\
De qualquer maneira, o código do gerador \textit{polynomial} está bem explicado e tem entradas já feitas na pasta dele, é só rodar o programa com essa entrada e mandar para a entrada do \textit{main} que a saída está conforme o esperado. \\
Este gerador, o \textit{polynomial}, testa a interpolação de polinômios com a decomposição qr, recebendo polinômios numa base arbritária e pontos a serem avaliados com um erro aleatório. É esperado que a saída do programa seja parecida com o vetor $t$ de entrada.


%----------------------------------------------------------------------------------------

\end{document}
