%%%%%%%%%%%%%%%%%%%%%%%%%%%%%%%%%%%%%%%%%
% Thin Sectioned Essay
% LaTeX Template
% Version 1.0 (3/8/13)
%
% This template has been downloaded from:
% http://www.LaTeXTemplates.com
%
% Original Author:
% Nicolas Diaz (nsdiaz@uc.cl) with extensive modifications by:
% Vel (vel@latextemplates.com)
%
% License:
% CC BY-NC-SA 3.0 (http://creativecommons.org/licenses/by-nc-sa/3.0/)
%
%%%%%%%%%%%%%%%%%%%%%%%%%%%%%%%%%%%%%%%%%

%----------------------------------------------------------------------------------------
%	PACKAGES AND OTHER DOCUMENT CONFIGURATIONS
%----------------------------------------------------------------------------------------

\documentclass[a4paper, 11pt]{article} % Font size (can be 10pt, 11pt or 12pt) and paper size (remove a4paper for US letter paper)
\usepackage[utf8]{inputenc}
\usepackage[protrusion=true,expansion=true]{microtype} % Better typography
\usepackage{graphicx} % Required for including pictures

\usepackage{mathpazo} % Use the Palatino font
\usepackage[T1]{fontenc} % Required for accented characters
\linespread{1.05} % Change line spacing here, Palatino benefits from a slight increase by default

\makeatletter
\renewcommand\@biblabel[1]{\textbf{#1.}} % Change the square brackets for each bibliography item from '[1]' to '1.'
\renewcommand{\@listI}{\itemsep=0pt} % Reduce the space between items in the itemize and enumerate environments and the bibliography

\renewcommand{\maketitle}{ % Customize the title - do not edit title and author name here, see the TITLE block below
\begin{flushright} % Right align
{\LARGE\@title} % Increase the font size of the title

\vspace{50pt} % Some vertical space between the title and author name

{\large\@author} % Author name
\\\@date % Date

\vspace{40pt} % Some vertical space between the author block and abstract
\end{flushright}
}

%----------------------------------------------------------------------------------------
%	TITLE
%----------------------------------------------------------------------------------------

\title{\textbf{MAC0214 - RE}\\ % Title
Relatório Final} % Subtitle

\author{\textsc{Victor Sena Molero} % Author
\\{\textit{8941317}}} % Institution

\date{\today} % Date

%----------------------------------------------------------------------------------------

\begin{document}

\maketitle % Print the title section

%----------------------------------------------------------------------------------------
%	ESSAY BODY
%----------------------------------------------------------------------------------------

\section*{Atividade e Objetivos}
Como especificado no início do semestre. O objetivo da minha participação nesta matéria foi aprimorar minhas habilidades como programador competitivo realizando 100 horas de provas durante o período da matéria. Todas essas horas foram registradas com detalhe no blog disponível no endereço http://victorsenam.github.io/blog-mac0214/.

%------------------------------------------------

\section*{Contests}
Foram somadas 107 horas e 50 minutos de contests (provas) durante o período. Algumas provas foram individuais e outras em dupla ou trio. Ao todo, foram resolvidos 116 problemas durante este tempo (incluindo os resolvidos em time).

%------------------------------------------------

\section*{Codeforces}
Codeforces é um site muito importante para a programação competitiva que reúne programadores de todo o mundo para a realização de contests online durante todo o ano. Os membros deste site possuem uma classificação à partir de um rating, calculado pela performance nas competições oficiais realizadas por eles, além de um classificação entre primeira divisão (onde os contests são mais difíceis) e segunda divisão. Para alcançar a primeira divisão é necessário que o rating esteja acima de uma determinada marca.

Durante esta matéria, consegui me manter na primeira divisão além de alcançar o melhor rating que já tive até agora. Segue uma lista com todas as provas realizadas individualmente neste site que foram consideradas:
\begin{itemize}
    \item 8VC Elimintion Round [2h30]
    \item Codeforces Round \#323 [2h00]
    \item Codeforces Round \#345 [2h00]
    \item Codeforces Round \#333 [2h00]
    \item CROC 2016 - Elimination Round [2h00]
    \item IndiaHacks 2016 [2h00]
    \item VK CUP 2016 - Round 1 (Div. 1 Edition) [2h00]
    \item Codeforces Round \#286 [2h00]
    \item Codeforces Round \#349 [2h00]
    \item Codeforces Round \#351 (VK CUP Round 3, Div. 1 Edition) [2h00]
    \item Codeforces Round \#355 [2h00]
    \item Codeforces Round \#285 [2h00]
    \item Codeforces Round \#107 [2h00]
    \item Codeforces Round \#302 [2h00]
    \item Codeforces Round \#219 [2h00]
    \item Codeforces Round \#352 [2h00]
    \item Codeforces Round \#114 [2h00]
    \item Codeforces Round \#356 [2h00]
    \item Codeforces Round \#348 (VK CUP Round 2, Div. 1 Edition) [2h00]
    \item Codeforces Round \#353 [2h00]
    \item Codeforces Round \#347 [2h00]
    \item Codeforces Round \#368 [2h00]
\end{itemize}

%----------------------------------------------------------------------------------------

\section*{Time}
International Collegiate Programming Contest é o nome da maior competição de programação do mundo, organizada pela ACM anualmente. Várias fases classificatórias são realizadas pelo mundo para esta competição. Eu participo de um time que se prepara, em conjunto, para essas classificatórias. O nome do time é "O XOR é livre" e ele é formado por mim, Nathan Benedetto Proença e Gabriel de Russo e Carmo.

Para se preparar para as competições oficiais, nós realizamos algumas provas em conjunto. Segue a lista de todas as que foram consideradas para esta matéria. A nossa performance como time melhorou considerávelmente durante o semestre

\begin{itemize}
    \item Contest Organizado pelo técnico Renzo Gonzalo Gomez Diaz (https://www.maratonando.com.br/contest/56bbb77457fe25de1001dcd6) [3h30]
    \item LATAM 2013 (Latin American Regionals) [5h00]
    \item UKIEPC 2015 (UK and Ireland Programming Contest) [5h00]
    \item NCPC 2015 (Nordic Collegiate Programming Contest) [5h00]
    \item GCPC 2015 (Germand Collegiate Programming Contest) [5h00]
    \item ICPC 2012 (World Finals) [5h00, participei apenas durante 3h00]
    \item Singapore 2015 [5h00]
    \item Seletiva UFMG 2016 [5h00]
    \item IPSC 2016 (Internet Problem Solving Contest) [5h00]
\end{itemize}

\section*{Torneios}
Durante este tempo eu participei também de alguns torneios na internet. Torneios são competições com várias fases que tem critérios próprios de classificação.

Participar neste tipo de competição ajuda a treinar a prática de competições, que envolve se concentrar em fazer a prova e resistir à pressão de estar competindo em tempo real com participantes de alto nível.

Segue uma lista dos torneios considerados nesta matéria e o tempo somado por cada um deles.

\begin{itemize}
    \item Google CodeJam 2016 [5h00]
    \item Yandex Algorithm 2016 [3h30]
    \item CodeChef SnackDown 2016 (em dupla com Gabriel de Russo e Carmo) [6h00]
\end{itemize}

\section*{Outros}
Além dessas existem outras competições que foram realizadas durante este período. São elas:

\begin{itemize}
    \item Topcoder SRM 668 [1h30]
    \item CSAcademy Beta Round 4 [2h00]
\end{itemize}

\section*{Conclusão}
A matéria me ajudou a organizar melhor meus estudos para a maratona e o tempo investido cumpriu o objetivo de me fazer melhorar considerávelmente como competidor tanto em provas individuais quanto em provas com times.

\end{document}
