%%%%%%%%%%%%%%%%%%%%%%%%%%%%%%%%%%%%%%%%%
% Thin Sectioned Essay
% LaTeX Template
% Version 1.0 (3/8/13)
%
% This template has been downloaded from:
% http://www.LaTeXTemplates.com
%
% Original Author:
% Nicolas Diaz (nsdiaz@uc.cl) with extensive modifications by:
% Vel (vel@latextemplates.com)
%
% License:
% CC BY-NC-SA 3.0 (http://creativecommons.org/licenses/by-nc-sa/3.0/)
%
%%%%%%%%%%%%%%%%%%%%%%%%%%%%%%%%%%%%%%%%%

%----------------------------------------------------------------------------------------
%	PACKAGES AND OTHER DOCUMENT CONFIGURATIONS
%----------------------------------------------------------------------------------------

\documentclass[a4paper, 11pt]{article} % Font size (can be 10pt, 11pt or 12pt) and paper size (remove a4paper for US letter paper)
\usepackage[utf8]{inputenc}
\usepackage[protrusion=true,expansion=true]{microtype} % Better typography
\usepackage{graphicx} % Required for including pictures

\usepackage{mathpazo} % Use the Palatino font
\usepackage[T1]{fontenc} % Required for accented characters
\linespread{1.05} % Change line spacing here, Palatino benefits from a slight increase by default

\makeatletter
\renewcommand\@biblabel[1]{\textbf{#1.}} % Change the square brackets for each bibliography item from '[1]' to '1.'
\renewcommand{\@listI}{\itemsep=0pt} % Reduce the space between items in the itemize and enumerate environments and the bibliography

\renewcommand{\maketitle}{ % Customize the title - do not edit title and author name here, see the TITLE block below
\begin{flushright} % Right align
{\LARGE\@title} % Increase the font size of the title

\vspace{50pt} % Some vertical space between the title and author name

{\large\@author} % Author name
\\\@date % Date

\vspace{40pt} % Some vertical space between the author block and abstract
\end{flushright}
}

%----------------------------------------------------------------------------------------
%	TITLE
%----------------------------------------------------------------------------------------

\title{\textbf{MAC0214 - PR}\\ % Title
Apresentação do Projeto} % Subtitle

\author{\textsc{Victor Sena Molero} % Author
\\{\textit{8941317}}} % Institution

\date{\today} % Date

%----------------------------------------------------------------------------------------

\begin{document}

\maketitle % Print the title section

%----------------------------------------------------------------------------------------
%	ESSAY BODY
%----------------------------------------------------------------------------------------

\section*{Atividade e Objetivos}

O projeto proposto como Atividade de Cultura e Extensão é o treino para a Maratona de Programação, com foco na competição conhecida como ACM-ICPC. O objetivo do estudo é aprimorar meu conhecimento em algoritmos e técnicas usadas nas competições de programação visando ajudar meu time a alcançar uma boa performance na ICPC de 2016/2017.

%------------------------------------------------

\section*{Tarefas}

Para alcançar meu objetivo pretendo comparecer a tantos treinos do grupo MaratonIME quanto for possível, o grupo se reúne uma vez por semana e pode aumentar a frequência para duas, além de realizar várias competições de treino individuais e com meu time em online judges como Codeforces, Ahmed-Aly, Maratonando e OpenTrains.

%------------------------------------------------

\section*{Acompanhamento}

Pretendo colocar todas as informações sobre meu treino em um blog disponível em http://victorsenam.github.io/blog-mac0214/ chamado Road to MAC0214. Pretendo atualizá-lo todo fim de semana. \\
O site também contém um contador de quantas horas de contest foram registradas na secção Stats.

\section*{Contagem de Horas}
Pretendo contar como atividade somente horas de contest feitos por mim e horas de treinos onde eu, de fato, me dediquei a passar problemas. Tudo será documentado no blog. \\

\section*{Supervisor}
O meu supervisor será o coach da maratona Renzo Gonzalo Gomez Diaz.

%----------------------------------------------------------------------------------------

\end{document}
